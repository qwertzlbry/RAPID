\documentclass{article}


\usepackage{arxiv}

\usepackage[utf8]{inputenc} % allow utf-8 input
\usepackage[T1]{fontenc}    % use 8-bit T1 fonts
\usepackage{hyperref}       % hyperlinks
\usepackage{url}            % simple URL typesetting
\usepackage{booktabs}       % professional-quality tables
\usepackage{amsfonts}       % blackboard math symbols
\usepackage{nicefrac}       % compact symbols for 1/2, etc.
\usepackage{microtype}      % microtypography
\usepackage{lipsum}
\usepackage{graphicx}
\graphicspath{ {./images/} }
\usepackage{setspace}
\usepackage{subcaption}

%\usepackage{showframe}
\usepackage{makecell}
\usepackage{amsmath}
\usepackage{amssymb}
\usepackage{natbib}
\usepackage{soul}
\usepackage{tablefootnote}
\usepackage{hyperref}
\usepackage{yhmath}
\usepackage{threeparttable}
\usepackage{booktabs}
\usepackage{multirow}
\usepackage{pifont}
\usepackage{pdflscape}
\usepackage{float}
\usepackage[T1]{fontenc}
%\usepackage[a4paper]{geometry}
\usepackage{graphicx}
\usepackage{appendix}
\usepackage{caption}
\usepackage{algorithm}
\usepackage[noend]{algpseudocode}
\usepackage{booktabs}
\usepackage{siunitx}
\usepackage{longtable}
\usepackage{tabularx}
\usepackage{xcolor}


\title{RAPID: Risk of Attribute Prediction-Induced Disclosure in Synthetic Microdata}


\author{
 Matthias Templ \\
  School of Business\\
 Applied University of Science and Arts\\
 Northwestern Switzerland \\
  \texttt{matthias.templ@fhnw.ch} \\
  %% examples of more authors
   \And
 Oscar Thees \\
  School of Business\\
 Applied University of Science and Arts\\
 Northwestern Switzerland \\
  \texttt{oscar.thees@fhnw.ch} \\
  \And
Roman M\"uller \\
 School of Business\\
Applied University of Science and Arts\\
Northwestern Switzerland \\
\texttt{roman.mueller@fhnw.ch} \\
  %% \AND
  %% Coauthor \\
  %% Affiliation \\
  %% Address \\
  %% \texttt{email} \\
  %% \And
  %% Coauthor \\
  %% Affiliation \\
  %% Address \\
  %% \texttt{email} \\
  %% \And
  %% Coauthor \\
  %% Affiliation \\
  %% Address \\
  %% \texttt{email} \\
}

\begin{document}
\maketitle
\begin{abstract}
% Statistical data anonymization increasingly relies on fully synthetic microdata, for which traditional identity disclosure measures are less informative than an adversary’s ability to infer sensitive attributes. We propose a simple, model-agnostic inference disclosure metric for synthetic data. The attacker trains a predictive model on the synthetic data and scores individuals in the original data. For continuous targets, risk is the share of cases whose relative error falls below a tolerance; for categorical targets, we introduce a relative confidence score that normalizes the model’s probability for the true class by a class-specific baseline, and we summarize risk by the fraction above a user-chosen threshold. The metric is calibrated for class imbalance, easy to communicate, and compatible with any synthesizer or learning algorithm. We illustrate how to choose thresholds, estimate uncertainty, and compare synthesizers. \textcolor{red}{We also relate our measure to existing disclosure concepts and to recent implementations in synthetic-data toolkits.} Our results suggest the metric provides an interpretable upper bound on practical attribute-inference risk \textcolor{blue}{check: while aligning with modern intruder models.} 

Statistical data anonymization increasingly relies on fully synthetic microdata, for which classical identity disclosure measures are less informative than an adversary’s ability to infer sensitive attributes from released data. We introduce RAPID (Risk of Attribute Prediction–Induced Disclosure), a disclosure risk measure that directly quantifies inferential vulnerability under a realistic attack model. An adversary trains a predictive model solely on the released synthetic data and applies it to real individuals’ quasi-identifiers. For continuous sensitive attributes, RAPID reports the proportion of records whose predicted values fall within a specified relative error tolerance. For categorical attributes, we propose a baseline-normalized confidence score that measures how much more confident the attacker is about the true class than would be expected from class prevalence alone, and we summarize risk as the fraction of records exceeding a policy-defined threshold. This construction yields an interpretable, bounded risk metric that is robust to class imbalance, independent of any specific synthesizer, and applicable with arbitrary learning algorithms. We illustrate threshold calibration, uncertainty quantification, and comparative evaluation of synthetic data generators using simulations and real data. Our results show that RAPID provides a practical, attacker-realistic upper bound on attribute-inference disclosure risk that complements existing utility diagnostics and disclosure control frameworks.
\end{abstract}


% keywords can be removed
%\keywords{First keyword \and Second keyword \and More}


\section{Introduction}

% \textcolor{red}{NOTE:the functions for RAPID will be put to the already existing R package riskutility before submission (Matthias will do).}

% Open research data (ORD) are increasingly recognized as essential for scientific transparency and reproducibility \citep{2015_Nosek}. Nevertheless, the proportion of shared datasets remains low. For instance, only 23\% of projects funded by the Swiss National Science Foundation (SNSF) provide ORD \citep{SNF_ORD2024}. Legal constraints, such as data protection laws and usage restrictions, often impede data sharing. In addition, many researchers and data custodians are unaware of existing methods for data anonymization or synthetization. 
% % In fully synthetic datasets, the right to be forgotten becomes irrelevant, as no individual’s real data are included.

% The notion that synthetic data enable open science without compromising privacy is misleading. Even in synthetic datasets, adversaries may exploit statistical dependencies to infer sensitive information, highlighting the need for rigorous risk assessment \citep{Stadler20c}. Therefore, a critical challenge remains: how to rigorously and meaningfully quantify inference disclosure risk for synthetic data.

Open research data (ORD) are increasingly recognized as essential for scientific transparency and reproducibility \citep{2015_Nosek}. Yet the proportion of shared datasets remains low; for instance, only 23\% of projects funded by the Swiss National Science Foundation (SNSF) currently provide ORD [Swiss National Science Fund (SNSF), 2024]. Legal constraints and contractual usage restrictions often hinder the release of microdata, and the technical barrier to implementing modern anonymization and synthetic data workflows remains substantial for many research teams. This gap motivates the growing use of fully synthetic microdata—but it also raises an immediate question for data stewards: how should disclosure risk be quantified when the primary threat is no longer re-identification, but inference about sensitive attributes?

\subsection{The user perspective: data utility}

From the user’s perspective, synthetic data must satisfy two core requirements: statistical similarity to the original data and plausibility in terms of logical and structural integrity. Beyond reproducing marginal distributions and associations, synthetic data must respect domain-specific constraints, such as realistic household compositions, non-negative expenditures (e.g., on medication), internally consistent demographic characteristics (e.g., no underage individuals with adult children), and valid event sequences (e.g., a PhD obtained after a master’s degree). Violations of such constraints can severely limit the credibility and usability of synthetic datasets, even when distributional similarity is high.

Figure~\ref{fig:utility} illustrates a general workflow for generating synthetic data and assessing their fitness for use. Analysts typically aim to (i) answer predefined research questions by fitting statistical or machine learning models and (ii) explore the data to identify new patterns and relationships. These objectives are attainable only when the synthetic data approximate the original data not merely in distribution, but also in structure and internal coherence.

\begin{center}
    \begin{figure}[!htp]
        \centering
         \hspace*{-0.8cm}
        \includegraphics[width=1.1\linewidth]{flowchart_user_perspective.pdf}
        \caption{Ideal synthetic data generation workflow. A synthetic data generator is trained on original data to produce synthetic data that preserves 
statistical properties for downstream analysis.}
        \label{fig:utility}
    \end{figure}
\end{center}

Crucially, however, high analytical utility alone does not guarantee safe data release. From the data provider’s perspective, increasing utility often entails preserving strong dependencies among variables—precisely the information that may enable an adversary to infer sensitive attributes. Disclosure risk therefore does not necessarily decrease as synthetic data become more useful; in fact, it may increase. In practice, this tension is commonly conceptualized using Risk–Utility (RU) Maps \citep{Duncan01a} and more recent multivariate extensions \citep{Thees25pca}, which support informed decisions about acceptable trade-offs between analytical value and disclosure protection. Assessing inference disclosure risk in a way that is commensurate with modern, high-utility synthetic data remains a key methodological challenge.


\subsection{Overview of Synthetic Data Generation Methods}


Synthetic data offer a promising solution, both for facilitating open data sharing and for long-term archiving, where access rights must be preserved. Research on synthetic data has progressed considerably since the 1980s, notably following the seminal contribution of \citet{Rubin93}, who introduced the idea of replacing sensitive values with simulated (i.e., imputed) ones drawn from predictive distributions.

Today, most synthetic data are generated using machine learning (ML) or artificial intelligence (AI) methods. These fall broadly into two categories:

\begin{itemize}
    \item Conditional modeling approaches, where each variable is synthesized conditionally on others, often in a sequential fashion. Techniques include decision trees, random forests, gradient boosting (e.g., XGBoost), multiple regression, and more recently, large language models (LLMs). These methods are typically applied in a stepwise process where each variable is synthesized conditioned on previously generated ones. It is implemented sequentially and recursively. A few approaches \citep{simPop} allow for accommodating missing data, complex survey designs, clustering, and hierarchical structures. Parameter fitting is performed on the original, non-anonymized data, and synthetic values are drawn from estimated conditional distributions.
    \item Joint modeling approaches, which attempt to model the entire joint distribution of all variables simultaneously. This class includes generative adversarial networks (GANs) and other deep generative models. Joint modeling methods generally require large training datasets and especially for synthetic data generation \citep{mekonnen24}, careful tuning of hyperparameters \citep{miletic24}, and significant computational resources. Moreover, they may struggle with outliers \citep{Stadler20c}, weakly correlated data structures (which are common in practice) \citep{ward25}, and learning intricate relationships among variables \citep{thees24}. 
\end{itemize}

Note that conditional GAN's to create synthetic data are also joint modelling approaches, and conditional here means the generation of full synthetic records conditioned on some known context or attributes. The generation is not sequential per variable.

Comparative evaluations suggest that only a small number of methods currently achieve high utility for synthetic data, particularly in complex tabular settings \citep{thees24}.



% How well an intruder can predict sensitive information given the synthetic data provided?

% Disclosure risk of synthetic data might be high. There is simple no privacy guarantee of synthetic data, even if methods from differential privacy are applied, because they does not account for inference attribute disclosure, the main source of risk for synthetic data. 

% One of the main challenges for data anonymization is to balance the risk-utility tradeoff properly ~\citep{Hundepool2012}. There are plenty of methods that are used in the traditional SDC approach, namely Risk-Utility maps~\citep{duncan11}, $k$-anonymity~\citep{Samarati98}, or $l$-diversity~\citep{Machanava07} and variants of it. Since synthetic data severs the connection between individuals and their data, an intruder can't be sure anymore if the value of a sensitive variable is true in the synthetic data. It is possible for a random data generator to produce accurate information about a person by chance. However, since there is no way to distinguish which data points are true / correct and which are not, the presence of such matches is not helpful to an attacker. Hence concerns about re-identification risk are less relevant. Instead, attribution risk becomes the more pertinent issue~\citep{taub18}. 
% In this section, we assess the extent to which synthetic data are vulnerable to attribute inference attacks, where an adversary aims to deduce sensitive or confidential attributes of individuals using only the synthetic data \citep{kwatra24, barrientos18}. If a model trained on the synthetic data can accurately predict sensitive variables based on non-sensitive attributes, this indicates a meaningful risk of inferential disclosure.

\subsection{Disclosure risk measures}

\color{red}
%OLD VERSION:

%Synthetic data fundamentally alter the disclosure landscape. Because synthetic records are not direct replicas of real individuals, traditional re-identification risks, such as linkage attacks based on quasi-identifiers, are substantially reduced. Even if a synthetic record closely resembles a real individual, there is no reliable way for an intruder to verify whether the match is genuine or coincidental. As a result, re-identification risk becomes less meaningful in fully synthetic settings.

%The more relevant concern is \textcolor{red}{inferential disclosure: the risk that an adversary could accurately infer a sensitive attribute from other, non-sensitive variables in the synthetic release. This form of inferential disclosure} becomes particularly problematic when strong dependencies among variables are preserved. If a model trained solely on synthetic data can reliably predict confidential attributes in the real data, the synthetic release carries residual privacy risk \citep{taub18, barrientos18, kwatra24}.

%Because synthetic data are not literal copies of real individuals, they weaken direct record-level linkage, shifting the disclosure concern from re-identification to attribute disclosure -- that is, the risk that an attacker correctly infers a sensitive attribute from available non-sensitive variables. In official statistics, this distinction is formalized through taxonomies that separate identification, attribute, and inferential disclosure, as well as through intruder models such as the archive attacker, who possesses quasi-identifiers and seeks to infer confidential values.


%Classical disclosure measures designed to detect record-level proximity or exact value matches therefore provide limited guidance in this setting, motivating risk measures that directly target inference of sensitive attributes.

%For synthetic microdata, a realistic adversary is one who trains a predictive model on the released synthetic file and applies it to their own information. Figure~\ref{fig:prediction} illustrates this, i.e., how an attacker predicts sensitive information using trained parameters from released synthetic data. This paper formalizes this scenario and introduces a compact, attacker-realistic risk measure that directly answers the question:
%How often would a capable intruder be confidently correct about a sensitive attribute?




% A holdout dataset, as proposed by \cite{hittmeir20,PlatzerReutterer2021Holdout} is unlikely to adequately account for the population, as results heavily depend on how much of the population is captured in this holdout dataset (i.e., its size). Also, the metrics’ interpretation can be difficult when synthetic records that are identified as replicated uniques in terms of their QIs differ in sensitive attributes from the training records. This comes back to the issue of meaningful identity disclosure claims as discussed in the context of record-level similarity. Importantly, vulnerability can still be unacceptably high for non-unique records. Thus, uniqueness metrics can only provide a lower bound in terms of identity disclosure vulnerability.

% \textcolor{red}{Move to subsection 1.5?}Because our threat model assumes an intruder with access only to the released synthetic file, we do not adopt \emph{holdout}-style evaluations that require auxiliary real data; such methods hinge on the availability and representativeness of the holdout sample, and their conclusions can vary substantially with its size and coverage \citep{hittmeir20,PlatzerReutterer2021Holdout}. Instead, \textsc{RAPID} quantifies attribute-inference vulnerability under this more practical attacker constraint, yielding a model-agnostic, baseline-normalized risk measure that is comparable across releases without depending on external data.

% Modern machine-learning literature has sharpened similar risks under the labels model inversion and membership/attribute inference. These works show that high-utility predictive models can leak information about training labels or attributes, especially when confidence scores are used naively. Our formulation adapts that intuition to the release-and-analyze setting of official statistics, where the attacker’s model is trained only on synthetic data and evaluated on the original microdata. 


%Canonical SDC distinguishes (i) identity disclosure (i.e., linking a record to a specific individual); (ii) attribute disclosure (i.e., correctly learning sensitive values); (iii) membership disclosure (i.e., learning whether an individuals data is part of a dataset); and (iv) table disclosure (i.e., revealing confidential information through aggregate statistics) \citep{hundepool2012sdc}.

%For fully synthetic microdata, identity disclosure risk is typically low by design \citep{templ14risk,emam20}, as synthetic records are not literal representations of individuals. As a result, most attention shifts toward analytical validity and attribute disclosure risk.


%Existing attribute-disclosure diagnostics include:

%\begin{itemize}
 %   \item Match-based measures, which evaluate how often synthetic values exactly match original values \citep{taub18};
  %  \item Model-based measures, which train a predictive model on synthetic data and assess its performance (e.g., accuracy, sensitivity, or class-specific metrics) on the original labels \citep{hittmeir20}.
%\end{itemize}


%Modern synthetic-data toolkits increasingly include routines for evaluating attribute disclosure risk. One example is DiSCO \citep{Raab_2025_Practical}, which assesses risk based on the co-occurrence of quasi-identifiers and sensitive attributes in original and synthetic data. DiSCO quantifies an attacker's potential success by checking whether a combination of quasi-identifiers in the original dataset also appears in the synthetic dataset with the same sensitive value, thus offering a form of confidence-based evaluation. However, such approaches may still be sensitive to class imbalance and do not directly model the inference process itself.

%Modern synthetic-data toolkits have begun to expose routines for attribute disclosure risk. One example is DiSCO \citep{Raab_2025_Practical}, \textcolor{blue}{which implements model-based comparisons with reasonable defaults. However, such tools often fail to account for class imbalance and typically do not quantify how confident an attacker could be in a correct inference.} 

%\textcolor{red}{RM: DiSCO is not model-based and does quantify an attacker’s level of confidence. DiSCO evaluates whether a given quasi-identifier (QI) combination in the original data also appears in the synthetic dataset. If so, it checks whether the associated sensitive attribute in the synthetic data (in cases where 1-diversity) matches the true value in the original. DiSCO is then reported as the percentage of such cases relative to the number of rows in the original dataset.}

%Synthetic microdata mitigate the risk of direct record linkage because the released records are not literal copies of actual respondents. The classical statistical disclosure control (SDC) literature distinguishes three forms of disclosure: identification, attribute, and inferential disclosure \citep{Duncan_1989_Risk, Palley_1987_Use}. In this taxonomy, attribute disclosure arises when a sensitive value becomes known because it is logically deducible from the released data. Inferential disclosure, by contrast, refers to situations in which the data allow a user to infer new information about a respondent, even if the respondent has no exact counterpart in the released file and even if the inferred value is inexact \citep[p.~208]{Duncan_1989_Risk}. This notion builds on Dalenius's classical definition:
%\begin{quote}
%“If the release of the statistic $S$ makes it possible to determine the value [sensitive attribute] more accurately than is possible without access to $S$, a disclosure has taken place.” \citep[p.~432]{Dalenius_1977_Methodology}
%\end{quote}
%As \citet[p.~37]{Hundepool2012} note, inferential disclosure is rarely used in microdata practice and remains conceptually distinct from attribute disclosure. In many official microdata products, such as Scientific Use Files (SUFs), this concept is seldom operationalized, because enabling statistical inference is precisely the purpose of the release. Meanwhile, recent work in the synthetic-data and machine-learning literatures often uses attribute inference more broadly to describe any prediction of sensitive attributes (e.g., \citealt{barrientos18}). Although both traditions concern the learning of sensitive attributes, the classical notion is based on logical deducibility, whereas the modern machine-learning notion is probabilistic and model-based. In this paper, we follow the classical SDC tradition and maintain a clear distinction between attribute and inferential disclosure.

\color{black}
Classical SDC distinguishes (i) identity disclosure (i.e., linking a record to a specific individual); (ii) attribute disclosure (i.e., correctly learning sensitive values); and (iii) membership disclosure (i.e., learning whether an individual's data is part of a dataset) \citep{hundepool2012sdc}.

For fully synthetic microdata, identity disclosure risk is typically low \citep{templ14risk,emam20}, as synthetic records are not literal representations of individuals. As a result, most attention shifts toward analytical validity and attribute disclosure risk.

Existing attribute-disclosure diagnostics include:

\begin{itemize}
    \item Match-based measures, which evaluate how often synthetic values exactly match original values \citep{taub18};
    \item Model-based measures, which train a predictive model on synthetic data and assess its performance (e.g., accuracy, sensitivity, or class-specific metrics) on the original labels \citep{hittmeir20}.
\end{itemize}

One example for a match-based measure is DiSCO \citep{Raab_2025_Practical}, which assesses risk based on the co-occurrence of quasi-identifiers and sensitive attributes in original and synthetic data. DiSCO quantifies an attacker's potential success by checking whether a combination of quasi-identifiers in the original dataset also appears in the synthetic dataset with the same sensitive value, thus offering a form of confidence-based evaluation. However, such approaches may still be sensitive to class imbalance and do not directly model the inference process itself.

For model-based measures, a possible attacker scenario is one where an attacker trains a predictive model on the released synthetic datafile and applies it to their own information. Figure \ref{fig:prediction} illustrates this, i.e., how an attacker predicts sensitive information using learned model parameters based on released synthetic data. If a model trained solely on synthetic data can reliably predict confidential attributes in the real data, the synthetic release carries privacy risk \citep{taub18, barrientos18, kwatra24}. 

Several existing approaches operationalize this threat model in different ways. Early work by \citet{Palley_1987_Use} demonstrated that sensitive attributes can be predicted even from aggregate query responses. More recently, attribute-inference attacks \citep[e.g.,][]{Stadler20c} compare predictions from models trained on synthetic versus original data to assess relative risk. \textcolor{red}{Hittmeier reinehmen}

Bayesian risk measures instead formalize the intruder’s uncertainty directly: an intruder who knows all records except one updates a posterior distribution over the unknown sensitive value given the released synthetic data and knowledge of the data generation mechanism, treating high-posterior values as plausible guesses \citep{Reiter_2014_Bayesiana, Latner_2025_Buyer}.

Some methods additionally rely on holdout samples of real data for linkage-based evaluation or prediction validation \citep{SOURCE}.

This paper formalizes this scenario and introduces a compact, attacker-realistic risk measure that directly answers the question: How often would a capable intruder be confidently correct about a sensitive attribute?

The risk measure proposed here focuses on assessing inferential disclosure risk arising from the release of anonymized datasets, including both synthetic data and data anonymized using traditional methods. Inferential disclosure -- also referred to as predictive disclosure \citep{Willenborg_2001_Elements} -- occurs when the publication of microdata enables more accurate or more confident inferences about sensitive attributes than would have been possible in the absence of such data \citep{Duncan_1989_Risk, Hundepool2012}. This concept builds on Dalenius's foundational definition of disclosure:
\begin{quote}“If the release of the statistic $S$ makes it possible to determine the value [of a sensitive attribute] more accurately than is possible without access to $S$, a disclosure has taken place.” \citep[p.~432]{Dalenius_1977_Methodology}
\end{quote}
In practice, inferential disclosure arises when an attacker learns statistical patterns from released data -- using, for example, statistical or machine learning models -- and leverages these patterns to predict sensitive information.

While attribute disclosure risk measures such as $l$-diversity \citep{Machanavajjhala_2006_Ldiversity}, $\beta$-likeness \citep{Cao_2012_Publishing}, or, in the context of synthetic data, TCAP \citep{Taub_2019_Creating} and DiSCO \citep{Raab_2025_Practical}, assess the risk of revealing sensitive attributes for individuals whose records are contained in the original dataset, inferential disclosure can also affect individuals who were not part of the original dataset. Moreover, it may concern sensitive information of which the affected individual is not even aware. For example, an attacker may observe that the published data reveal a strong association between certain lifestyle indicators (e.g., diet, physical activity, and neighborhood characteristics) and an elevated risk of developing a chronic disease. Consequently, the attacker could infer that a similar individual outside the dataset faces a heightened disease risk, even in the absence of a diagnosis.

Although in many contexts -- such as the release of research data -- it is often neither feasible nor desirable to eliminate inferential disclosure risk entirely, as doing so would require destroying all meaningful relationships between sensitive and non-sensitive variables \citep{Dwork_2010_Difficulties}, we argue that inferential risk warrants increased attention in the era of big data and artificial intelligence. According to \cite{Muhlhoff_2021_Predictive}: "An individual’s (or group’s) predictive privacy is violated if sensitive information about that individual is statistically estimated against their will or without their knowledge on the basis of data of many other individuals, provided that these predictions lead to differential treatment or decisions that affect anyone’s social, economic, psychological, physical, ... wellbeing or freedom." This perspective aligns with arguments that there is no morally relevant distinction between directly accessing someone's private information and inferring it through statistical methods when the predictions have comparable accuracy \citep{Munch_2021_Privacy}.

Given the increasing importance of inferential disclosure in modern data ecosystems, we require risk measures that can assess this threat at the record level. While existing model-based measures evaluate overall prediction accuracy and typically provide aggregate, dataset-level risk summaries, RAPID addresses this limitation by assessing inference risk at the record level. This granular approach identifies which specific individuals face elevated disclosure risk, enabling data custodians to implement targeted risk mitigation strategies rather than applying uniform protections that may unnecessarily degrade data utility. \textcolor{red}{MERGE THIS WITH 1.5 Contributions}


\subsection{Guarantee's with differential privacy?}

In statistical data dissemination, access models range from Public Use Files (PUFs) to Scientific Use Files (SUFs), data enclaves, and query-only access systems. Each level reflects a different balance of accessibility, utility, and privacy risk -- and each has distinct implications for privacy-preserving techniques such as differential privacy (DP).

While DP remains central in interactive systems for releasing aggregate statistics, its applicability to synthetic tabular data releases is contested. Recent critiques argue that DP offers limited practical protection outside controlled query interfaces \citep{domingo21,BlancoJusticia2022,muralidhar23,DomingoFerrer_2025_Statistical} and can foster misinterpretations of disclosure risk \citep{MuralidharRuggles2024}.

Differential privacy provides a formal guarantee of output stability: it ensures that the probability of any particular output does not change much -- specifically, by more than a multiplicative factor $\exp(\varepsilon)$ -- when a single individual’s data is added to or removed from the dataset. This inclusion-exclusion stability is independent of the intruder’s background knowledge and does not rely on any assumptions about the data distribution or attacker capabilities.

This guarantee is particularly effective at preventing membership inference attacks, where the adversary tries to determine whether a specific individual was present in the dataset. If a synthetic dataset is generated using a differentially private mechanism, the attacker cannot confidently tell whether any one individual influenced the outcome.

However, attribute inference attacks exploit a different mechanism: rather than asking “Was this person in the dataset?”, the attacker asks “What is this person’s sensitive attribute, given the released data and what I know about them?” Differential privacy does not directly limit the accuracy of such inferences.

Why? Because DP only guarantees that whatever inferences are possible with an individual’s data would also have been approximately possible without it. That is, the risk does not stem from participation, but rather from population-level patterns that the synthetic data might preserve. If there are strong correlations in the population -- e.g., between age, education, and disease status -- then a synthetic dataset generated under DP may still encode those relationships. As a result, an attacker can accurately infer a sensitive attribute (e.g., health status) from non-sensitive ones (e.g., age and income), not because a specific person was included, but because the overall model captures a real dependency.

Thus, DP guarantees that synthetic data do not reveal whether someone was in the training data -- but not that sensitive attributes cannot be learned from non-sensitive ones. This distinction is crucial when assessing attribute disclosure risk in public-use synthetic microdata.

As a result, even DP-certified synthetic data may exhibit high attribute-inference risk. For this reason, differential privacy should be complemented with explicit, scenario-based disclosure risk assessments, particularly when synthetic data are intended for public release.

\color{blue}
RM: If I understand DP correctly, it does not claim that an attacker cannot learn sensitive information about an individual; rather, it guarantees that the attacker’s information gain is essentially the same whether or not a specific individual’s data are included in the dataset. The differential privacy guarantee applies not to the dataset itself, but rather to the data query mechanism. The rationale is that if it does not matter whether an individual was or was not included in the dataset, then no privacy breach can occur. Of course, this is highly debatable (see, for example, \href{https://link.springer.com/article/10.1007/s10676-021-09606-x}{Predictive privacy}). However, traditional methods would make the same claim; therefore, the critique presented in this section is not specific to DP. I would shorten this section introduce this problem more generally. 




\color{black}

% \paragraph{Relation to membership-disclosure and DP baselines.} \textcolor{red}{Proofread if RAPID should be already mentioned here.}
% Recent work on \emph{membership disclosure} frames attribute-inference risk through a differential-privacy lens by comparing two anonymized datasets that differ only in the presence of a single target individual---a ``member'' and a ``non-member'' version---and quantifying the \emph{incremental} improvement in an attribute-inference attack when the target is included \citep[e.g.,][]{francis2025betterattributeinferencevulnerability}. The goal there is to estimate per-person \emph{privacy loss due to inclusion}, anchored in DP's stability principle.

% \textcolor{red}{Move to next subsection?}This is not the focus of RAPID. Our contribution targets \emph{dataset-level attribute inference disclosure} for fully synthetic microdata under a realistic intruder who trains only on the released synthetic file and scores the original covariates. RAPID answers: \emph{How often could a capable attacker be confidently correct about a sensitive attribute, given only the synthetic release?} Methodologically, RAPID (i) does not require constructing counterfactual member/non-member datasets, (ii) does not attempt to attribute causal privacy loss to any individual, and (iii) normalizes model confidence by a class-specific baseline to yield interpretable, class-imbalance–robust risk summaries.

% Conceptually, the two lines of work address different questions:
% \begin{itemize}\setlength\itemsep{0.25em}
% \item \textbf{Membership-DP approaches:} per-individual \emph{inclusion risk}; requires paired member/non-member anonymized datasets; suited to certifying DP-style guarantees.
% \item \textbf{RAPID:} population-level \emph{attribute-inference propensity} from the released synthetic file; requires only the public synthetic data and original covariates for scoring; suited to statistical-agency risk–utility assessment and comparison across synthesizers.
% \end{itemize}
% RAPID complements, rather than replaces, membership-focused analyses: it provides an operational, model-agnostic upper bound on practical attribute-inference risk, while DP-style member/non-member comparisons quantify stability of the release with respect to any single person's inclusion.

% ZXXX

\paragraph{Relation to membership disclosure and DP baselines.}

Recent work on \emph{membership disclosure} situates attribute-inference risk within a differential privacy framework by comparing two anonymized datasets that differ only in the presence of a single individual, a “member” and a “non-member” version, and measuring the marginal improvement in inference accuracy when the individual is included \citep[e.g.,][]{francis2025betterattributeinferencevulnerability}. These approaches aim to quantify per-person \emph{privacy loss due to inclusion}, grounded in DP's stability guarantee.

This line of work addresses a different question than ours. While membership-based evaluations focus on individual-level stability, our focus lies on population-level vulnerability to attribute inference in synthetic microdata. The next subsection introduces a complementary disclosure risk measure, \textsc{RAPID}, which targets dataset-level inference propensity under a realistic intruder model, where the attacker has access only to the released synthetic file.

Rather than attributing risk to individuals, such methods aim to summarize how often synthetic data would enable accurate and confident inference of sensitive attributes—regardless of any specific person's presence. Both approaches contribute to understanding disclosure risk, but they serve distinct purposes: membership disclosure is suited to certifying DP-style guarantees, while inference-oriented measures such as \textsc{RAPID} support statistical-agency assessments, especially of public-use synthetic data.


% \subsection{Inference disclosure risk for synthetic data}\label{sec:inference-risk}

% Synthetic microdata mitigate record linkage because released records are not literal copies of respondents, shifting the attacker’s objective from identification to \emph{attribute (inference) disclosure}: correctly learning a confidential attribute from available non-sensitive information. This focus is consistent with the classical taxonomy separating identification, attribute, and inferential disclosure in official statistics \citep{Hundepool2012} and with practical guidance for synthetic releases that emphasizes attribution risk over reidentification risk \citep{taub18}. Closely related notions appear in the machine-learning literature under \emph{model inversion} and \emph{attribute inference} attacks \citep[e.g.,][]{barrientos18}.

% Holdout-based evaluations, such as those proposed by \citet{hittmeir20} and \citet{PlatzerReutterer2021Holdout}, assess disclosure risk by comparing synthetic records to an auxiliary real dataset. However, this approach has several limitations. Its conclusions depend heavily on the size and representativeness of the holdout sample, which may not adequately reflect the full population. Moreover, its metrics can be difficult to interpret, particularly when synthetic records match real individuals on quasi-identifiers but differ in sensitive attributes. Such discrepancies complicate the interpretation of identity disclosure and raise questions about what constitutes meaningful similarity. Crucially, even non-unique records may still be vulnerable to inference attacks, meaning that uniqueness metrics offer, at best, a lower bound on disclosure risk.

% Because our threat model assumes an intruder with access only to the released synthetic file -- and no holdout or auxiliary data -- we do not adopt holdout-style evaluations. Instead, our proposed measure, RAPID, directly quantifies attribute inference risk under this more realistic constraint. It is model-agnostic, normalized for class imbalance, and does not rely on external datasets, making it applicable and comparable across different releases and synthesizers.

% Closely related concepts from the machine learning literature include model inversion and attribute inference attacks. These approaches show that high-utility models can inadvertently reveal sensitive information about the training data -- especially when confidence scores are interpreted naively. Our formulation adapts these ideas to the release-and-analyze setting common in official statistics, where an attacker trains a model solely on synthetic data and evaluates it on real covariates to infer confidential values.

\subsection{Contributions}

This paper makes the following contributions.

\begin{enumerate}
\item \textbf{A new disclosure risk functional for synthetic data under a realistic threat model.}  
We propose RAPID (Risk of Attribute Prediction–Induced Disclosure) as a population-level measure \textcolor{red}{(RM: what does this mean?)} of attribute-inference disclosure risk for fully synthetic microdata. Our threat model assumes an intruder who has access only to the released synthetic dataset and to quasi-identifiers, but no auxiliary real data \textcolor{red}{(RM: Why does auxiliary data matter?)}, reflecting the conditions faced by external analysts of public-use synthetic files. RAPID directly operationalizes the attacker’s objective: confidently inferring sensitive attributes of real individuals using models trained solely on synthetic data. \textcolor{red}{While RAPID quantifies attribute inference in the modern, prediction-based sense, this vulnerability corresponds to \emph{inferential disclosure} in the classical SDC taxonomy}. By design, RAPID departs from re-identification-centric and table-based diagnostics by evaluating predictive inference success under a train-on-synthetic, infer-on-real setting.

\item \textbf{Baseline-normalized confidence scoring for categorical attributes.}  
We introduce a normalized gain measure that compares an attacker’s predicted probability for the true class to a class-specific baseline determined by class prevalence. This yields a calibrated notion of confidence beyond chance that explicitly accounts for class imbalance and avoids overstating risk in skewed distributions. By focusing on confidence-adjusted correctness rather than raw accuracy, RAPID captures the extent to which synthetic data enable \emph{meaningful} inference about sensitive attributes.

\item \textbf{A unified inference-risk framework for categorical and continuous sensitive attributes.}  
RAPID provides a consistent formulation for both discrete and continuous confidential variables. For categorical attributes, inference risk is quantified via baseline-normalized prediction confidence, while for continuous attributes it is defined through tolerance-based relative prediction error. This unified framework allows inference risk to be assessed consistently across categorical and continuous attributes, without discretizing continuous outcomes, introducing arbitrary \textcolor{red}{binning schemes}, or relying on auxiliary real or \textcolor{red}{counterfactual datasets}.

\item \textbf{Threshold-based, policy-interpretable risk summaries.}  
By summarizing disclosure risk as the proportion of records exceeding a confidence or accuracy threshold, RAPID yields interpretable, bounded metrics that are easy to communicate and tunable to institutional or regulatory risk tolerances. This formulation aligns naturally with risk–utility decision frameworks commonly used in statistical disclosure control and supports consistent comparison across different synthesizers, parameter settings, and data releases.

\item \textbf{Record-level risk indicators enabling attribution and diagnostic analysis.}  
Although RAPID is reported as a population-level disclosure metric, it is constructed from per-record risk indicators. This design enables granular analyses of inference vulnerability, including the identification of high-risk records, subgroup-specific risk patterns, and combinations of quasi-identifiers that disproportionately contribute to disclosure risk. We demonstrate how these record-level signals support attribution analyses and diagnostic evaluations that help data curators understand and mitigate the drivers of inference risk in synthetic data.

\item \textbf{Empirical validation, robustness analysis, and optional holdout-based evaluation.}  
Through simulation studies and real-data illustrations, we demonstrate how RAPID scales with dependency strength between quasi-identifiers and sensitive attributes, how threshold choice affects risk classification, and how results remain stable across a range of reasonable attacker models. In addition to its primary train-on-synthetic, infer-on-real evaluation protocol, RAPID can optionally be combined with holdout-based assessments when auxiliary real data are available, supporting internal benchmarking and synthesizer comparison without altering the core threat model.

\item \textbf{Practical implementation for disclosure control workflows.}  
RAPID is implemented in open-source software and integrates seamlessly with existing synthetic data generation pipelines. The implementation includes attacker-model ensembles, optional holdout-based evaluations, and diagnostic analyses that guide targeted risk mitigation. By focusing on population-level vulnerability rather than per-individual privacy loss, RAPID complements formal privacy frameworks such as differential privacy and is particularly suited to the evaluation of public-use synthetic microdata releases.
\end{enumerate}

% OLD:


% Our approach adopts a different threat model: we assume an intruder has access only to the released synthetic file and no auxiliary real data. This setting reflects the conditions faced by an external data analyst who receives a public-use synthetic dataset. In this scenario, we propose a new disclosure risk measure -- RAPID (Risk of Attribute Prediction–Induced Disclosure) -- which quantifies how often a capable attacker could be confidently correct about a sensitive attribute, based solely on synthetic data and known quasi-identifiers. Although RAPID quantifies attribute inference in the modern sense, this prediction-based vulnerability corresponds to inferential disclosure in the classical SDC taxonomy.

% The RAPID framework formalizes attribute-inference vulnerability as a function of prediction accuracy and prediction confidence. It is model-agnostic and does not require constructing counterfactual datasets, attributing risk to individuals, or certifying  compliance with formal privacy guarantees. Instead, it provides interpretable, normalized summaries of inference risk that account for class imbalance and allow for consistent comparison across different synthetic data releases and synthesizer settings.

% In contrast, recent work on membership disclosure -- often situated in the differential privacy literature -- focuses on quantifying the privacy loss attributable to the inclusion of an individual. This is typically achieved by comparing anonymized datasets that differ only by the presence or absence of a target individual and analyzing the marginal change in inference performance. While this approach aligns with the theoretical objectives of differential privacy, it addresses a fundamentally different problem: the stability of outputs with respect to individual participation, rather than the vulnerability of released data to sensitive-attribute inference.

% In sum, RAPID complements rather than replaces these formal privacy analyses. It provides a practical, attacker-aware measure of disclosure risk under realistic conditions and helps agencies evaluate whether synthetic data releases expose meaningful inferential vulnerabilities. Its emphasis on population-level risk, rather than per-individual privacy loss, makes it especially suitable for public-use synthetic microdata dissemination.

% \textcolor{blue}{RM: Maybe interesting: \\
% In the first preprint version of "Synthetic data – Anonymisation groundhog day" Stadler et al. (2020) published "Synthetic Data -- A Privacy Mirage" (see \url{https://arxiv.org/abs/2011.07018v1}). On page 6 they describe an attribute inference attack on synthetic data via a regression model. They train two LM's: one on the synthetic and the other on the original data. Then they compare the success rates of the two models for predicting the true sensitive value (also using a uncertainty thresholds). This way they can quantify how much the publishing of the synthetic data improves or reduces the prediction performance (i.e. inferential disclosure).}

% \subsection{Motivating example: training on synthetic, inferring on real data}
% \label{sec:motivation-attack}

% To illustrate the kind of inference threat we aim to quantify, we conducted a simple ``attacker'' experiment using publicly available microdata. We first prepared a real dataset from the \texttt{eusilc} public-use file (package \texttt{simPop}), retaining individuals with strictly positive labour income and renaming selected variables to be more interpretable: \texttt{income} (gross income), \texttt{age}, \texttt{gender}, \texttt{region}, \texttt{citizenship}, \texttt{econ\_status}, and \texttt{marital\_status}. Minor data hygiene followed the practice an adversary could reasonably adopt: \texttt{age} (mislabeled as a factor in the source) was coerced to numeric and sporadic missing values in \texttt{age} were imputed using $k$NN (\texttt{VIM::kNN}, no imputation flags).

% From this curated real file, we then generated a single fully synthetic microdata set with \texttt{synthpop} (\texttt{syn}, defaults). Importantly, the synthetic set plays the role of the \emph{released} data: the attacker is assumed to see only this file and standard analysis tools. We fit two off-the-shelf random-forest models (\texttt{ranger}) \emph{on the synthetic data}:
% \begin{enumerate}
%   \item A classifier for the confidential attribute \texttt{marital\_status} using all other variables as predictors.
%   \item A regressor for the continuous confidential attribute \texttt{income}, again using the remaining variables.
% \end{enumerate}
% Mimicking an intruder who applies their synthetic-trained model to a population of interest, we then fed the \emph{real} covariates into the fitted models and scored:
% \begin{itemize}
%   \item For \textbf{categorical} \texttt{marital\_status}, we obtained class probability vectors and a confusion matrix against the true labels in the real file. Even without access to the original training labels, the model achieved non-trivial accuracy of 82\%, evidencing that a synthetic-trained classifier can correctly guess a sensitive category for many individuals.
%   \item For \textbf{continuous} \texttt{income}, we produced a scatter plot of predicted versus true values, a density of prediction errors, and a boxplot of relative errors (in \%) $100 \cdot \lvert \widehat{y}-y\rvert / y$ (Figure~\ref{fig:real-attack}). Visually, even many real records fall close to the 45-degree line, indicating that an attacker can often approximate incomes within a practically meaningful tolerance, the prediction quality isn't high and the relative prediction errors are considerable large.
% \end{itemize}

% \begin{figure}[t]
% \centering
% \includegraphics[width=\textwidth]{real_data_attack.pdf}
% \caption{Real-data attack using a model trained on a fully synthetic file. Left: predicted vs.\ true income with 45-degree reference; middle: distribution of prediction errors; right: distribution of relative errors. The attacker never sees the original labels during training.}
% \label{fig:real-attack}
% \end{figure}

% This stylized exercise shows that \emph{train-on-synthetic, test-on-real} can yield correct and, in many cases, \emph{confident} inferences about confidential attributes. However, standard summaries such as overall accuracy or mean error are poorly aligned with disclosure concerns. They (i) treat a 0.51 and 0.99 posterior for the correct class as equivalent, obscuring \emph{confidence} in correct guesses; (ii) are sensitive to \emph{class imbalance} and calibration, inflating risk in prevalent classes; and (iii) do not provide a stable, thresholded quantity that agencies can interpret and compare across releases and synthesizers.

% These limitations motivate our new risk measure, \textbf{RAPID} (Risk of Attribute Prediction–Induced Disclosure). RAPID evaluates the intruder scenario above but summarizes risk in an interpretable, policy-ready way: for categorical targets, it normalizes the model's record-level confidence for the true class by a class-specific baseline implied by the attacker's own model and reports the fraction of individuals exceeding a user-chosen confidence ratio; for continuous targets, it reports the fraction within a specified relative-error band. In short, RAPID focuses on \emph{how often an attacker could be confidently right}, while remaining robust to base rates and comparable across synthesizers and tasks. 


% XXXX

\subsection{Motivating Example: Training on Synthetic, Inferring on Real Data}
\label{sec:motivation-attack}

To illustrate the type of inference threat that motivates our proposed risk measure, we conducted a simple attacker-style experiment using publicly available microdata. We used the \texttt{eusilc} public-use file provided via the \texttt{simPop} package \textcolor{red}{SOURCE} in R, retaining individuals with strictly positive labour income. Selected variables were renamed for interpretability, including \texttt{income} (gross income), \texttt{age}, \texttt{gender}, \texttt{region}, \texttt{citizenship}, \texttt{econ\_status}, and \texttt{marital\_status}. Minor data cleaning mimicked steps a plausible adversary might perform: \texttt{age}, which was mislabelled as a factor, was coerced to numeric, and sporadic missing values in \texttt{age} were imputed using $k$-nearest neighbours via the \texttt{VIM::kNN} \textcolor{red}{SOURCE} function without adding imputation flags.

From this curated real dataset, we generated a single fully synthetic microdata file using the \texttt{synthpop} package \citep{Nowok_2016_Synthpop} with default settings. This synthetic file represents the public release; the attacker is assumed to have access only to this file and standard analysis tools. On the synthetic data, we trained two random forest models (using the R package \texttt{ranger}) \textcolor{red}{SOURCE}: one classifier to predict the categorical sensitive variable \texttt{marital\_status}, and one regressor to estimate the continuous sensitive variable \texttt{income}, both using all other variables as predictors.

Mimicking an intruder who applies the trained model to a real population of interest, we then input the real covariate values into the synthetic-trained models. The results reveal two key risks. For \texttt{marital\_status}, the classifier achieved a prediction accuracy of 82\% on real data, despite never having seen true labels of the original data during training. This suggests that a synthetic-trained model can often correctly guess a sensitive category. For \texttt{income}, predicted values were compared to true values via scatter plots, error distributions, and relative error summaries (Figure~\ref{fig:real-attack}). Although absolute prediction quality was moderate, many points lay close to the 45-degree line, indicating that real individuals’ incomes could often be approximated within a practically meaningful range.

\begin{figure}[t]
\centering
\includegraphics[width=\textwidth]{real_data_attack.pdf}
\caption{Real-data attack using a model trained on a fully synthetic file. Left: predicted vs.\ true income with 45-degree reference; middle: distribution of prediction errors; right: distribution of relative errors. The attacker never sees the original labels during training.}
\label{fig:real-attack}
\end{figure}

This stylized exercise highlights that a train-on-synthetic, test-on-real setup can lead to correct and, in many cases, \emph{confident} inferences about confidential attributes. Standard summary metrics such as accuracy or mean squared error are insufficient in this context. They fail to capture the adversary’s confidence in correct guesses, treat low and high posterior probabilities for correct predictions equally, and are highly sensitive to class imbalance. These limitations make such metrics ill-suited for assessing disclosure risk in synthetic data.

To address these issues, we introduce \textbf{RAPID} (Risk of Attribute Prediction–Induced Disclosure). RAPID evaluates the same attacker scenario but summarizes the risk in a more interpretable and policy-relevant way. For categorical variables, it quantifies the share of individuals for whom the model’s confidence in the true class—normalized by a class-specific baseline—exceeds a chosen threshold. For continuous variables, it reports the fraction of records for which the predicted value lies within a specified relative error margin. In both cases, RAPID captures how often a capable attacker could be \emph{confidently right}, while remaining robust to base rates and allowing meaningful comparison across datasets, tasks, and synthesizers.

% \subsection{Problem and Motivation for the new measure}

% For fully synthetic microdata, record linkage is deliberately weakened because released records are not literal copies of respondents. The practical privacy concern therefore shifts from re-identification to \emph{attribute (inference) disclosure}: can an intruder, using only the synthetic release and standard analytical tools, correctly infer a confidential attribute about real individuals from non-sensitive variables?

% We adopt a realistic variant of the classic archive-attacker mindset: the adversary trains a predictive model on the synthetic file and then applies it to the original covariates. If the resulting predictions for the confidential attribute are frequently and \emph{confidently} correct, the release carries inferential disclosure risk. Existing summaries (e.g., simple accuracy or exact matches) have three shortcomings in this setting:
% \begin{enumerate}
%   \item \textbf{They ignore confidence.} Treating a barely-above-chance and a near-certain prediction as equivalent obscures the cases agencies worry about most.
%   \item \textbf{They are sensitive to class imbalance and calibration.} Raw probabilities and accuracies can overstate risk for common classes and understate it for rare ones.
%   \item \textbf{They lack comparability across releases and models.} Different synthesizers induce different base rates and probability scales, making side-by-side interpretation difficult.
% \end{enumerate}

% We therefore seek a risk summary that (i) emphasizes \emph{confident} correct inferences, (ii) adjusts for base-rate differences so minority classes are not penalized, (iii) is model-agnostic and easy to compute, and (iv) remains interpretable for stakeholders and comparable across releases. Our solution—described formally in Section~\ref{sec:new-measure}—implements a simple \emph{train-on-synthetic, test-on-original} protocol and aggregates per-record evidence using a thresholded, baseline-normalized notion of confidence for categorical attributes, and a tolerance-based notion of closeness for continuous attributes. Default attacker models, threshold choices, and uncertainty quantification are provided later together with implementation guidance (Algorithms~\ref{algo1}–\ref{algo2}).


% Our evaluation approach is conceptually related to the archive attacker model \citep{Hundepool_2012}, in that it measures how well sensitive attributes of real individuals can be inferred. However, unlike a full archive attacker, we assume the adversary has access only to the synthetic data, not the original dataset. We simulate this attack by training predictive models on the synthetic data $\mathbf{Z}^{(s)}$ and then evaluating their ability to recover sensitive attributes in the original data $\mathbf{Z}$.

% Existing attribute-disclosure summaries leave three gaps:

% \begin{description}
%     \item[Confidence blindness.] Accuracy treats a 0.51 and 0.99 posterior equally; agencies care about confident correct guesses.
%     \item[Class imbalance and heterogeneity.] A uniform thresh qold  on raw probabilities penalizes minority classes. Our baseline-normalized ratio r_i=g_i/b_i compares confidence for record i against the model’s typical confidence for that class, avoiding spurious high risk in rare categories.
%     \item[Comparability across releases.] Different synthesizers and tasks yield different marginal class distributions and calibration; a relative score makes cross-release interpretation stable.
% \end{description}
    
% Finally, the protocol mirrors the archive attacker mindset but restricts the intruder to the released synthetic file—arguably the most relevant practical case for fully synthetic microdata.


% We formalize a realistic intruder who has access only to the released synthetic file and standard analytical tools. Let the original data be \(\mathbf{Z}=[\mathbf{X},\mathbf{y}]\) with \(n\) records and the synthetic data be \(\mathbf{Z}^{(s)}=[\mathbf{X}^{(s)},\mathbf{y}^{(s)}]\) with \(n^{(s)}\) records. An attacker trains a predictive model on the synthetic data,
% \[
% \hat{\Theta}^{(s)} \leftarrow \mathcal{M}\!\big(\mathbf{y}^{(s)} \sim \mathbf{X}^{(s)}\big),
% \]
% and applies it to the original covariates to obtain predictions \(\hat{\mathbf{y}}=f(\mathbf{X};\hat{\Theta}^{(s)})\). If \(\hat{\mathbf{y}}\) can frequently and \emph{confidently} recover \(\mathbf{y}\), then the release leaks attribute information. This \emph{train-on-synthetic, test-on-original} protocol mirrors the archive-attacker mindset \citep{Hundepool_2012} while restricting the adversary to the public release.

% Two design choices are crucial for a meaningful risk summary. \textbf{(i) Attacker capability.} We assume a competent adversary and use a strong off-the-shelf learner (e.g., random forest or gradient boosting) as the default \(\mathcal{M}\). Agencies may take a conservative stance by repeating the evaluation over a small model suite and reporting the maximum risk across models. \textbf{(ii) Calibration for base rates.} Raw accuracy or unnormalized class probabilities can overstate risk in imbalanced settings. We therefore evaluate per-record \emph{confidence} relative to a class-specific baseline implied by the attacker’s model.

% Concretely, for a continuous sensitive variable \(y\in\mathbb{R}\), we flag record \(i\) as at risk when the relative prediction error is small,
% \[
% g_i \;=\; \mathbb{I}\!\left(\left|\frac{y_i-\hat{y}_i}{\hat{y}_i}\right|<\tau\right),
% \]
% and summarize risk by the share \(g=\frac{1}{n}\sum_{i=1}^n g_i\) (with tolerance \(\tau\) chosen by policy). For a categorical sensitive variable, let \(g_i=\Pr(\hat{y}_i=y_i\mid \mathbf{x}_i,\hat{\Theta}^{(s)})\) be the model’s probability assigned to the true class and let
% \[
% b_i \;=\; \tilde{p}_{y_i} \;=\; \frac{1}{n_{y_i}}\sum_{j:\,y_j=y_i}\Pr(\hat{y}_j=y_i\mid \mathbf{x}_j,\hat{\Theta}^{(s)}),
% \]
% be the model’s \emph{baseline} confidence for class \(y_i\) averaged over all real records in that class. The \emph{relative confidence score} \(r_i=g_i/b_i\) is thus \(>1\) when the model is more confident for record \(i\) than it typically is for that class. We then report the \emph{confidence rate above threshold}
% \[
% c_{\tau}\;=\;\frac{1}{n}\sum_{i=1}^n \mathbb{I}\{r_i>\tau\},\qquad \tau>1,
% \]
% which answers the policy question: ``For what fraction of people could an attacker be \emph{confidently} correct relative to class prevalence?'' Algorithms~\ref{algo1}–\ref{algo2} detail the full procedure.

% The resulting metric is (i) model-agnostic and easy to compute, (ii) interpretable for non-technical stakeholders, (iii) robust to class imbalance via baseline normalization, and (iv) comparable across synthesizers and releases. In practice, threshold \(\tau\) can be set by policy (e.g., \(\tau\in[1.1,1.5]\)) or calibrated via a simple permutation null on \(\mathbf{y}\) in \(\mathbf{Z}\); see Section~\ref{algo1} for implementation details and later sections for uncertainty quantification and how the measure complements standard utility diagnostics.


\section{Method: Quantifying Inference Risk via RAPID}

Our proposed measure, \textbf{RAPID} (Risk of Attribute Prediction–Induced Disclosure), quantifies how often an attacker trained on synthetic data can make confidently correct inferences about a sensitive attribute in the original dataset. This measure reflects a realistic threat model in which the attacker only has access to the released synthetic microdata, trains a predictive model, and applies it to target individuals whose quasi-identifiers are known to the attacker.

The key idea is to measure how much better the synthetic-trained model performs compared to a naive baseline strategy that ignores all quasi-identifier information. For categorical attributes, this baseline corresponds to guessing according to the marginal class frequencies; for continuous attributes, it corresponds to a reference error level. Comparing against such a baseline ensures the metric is robust to class imbalance and reflects true information gain from quasi-identifiers.



\subsection{Setup and Notation} 

 
Let the original microdata be denoted by
\[
\mathbf{Z} = [\mathbf{X}_Q,\; \mathbf{X}_U,\; \mathbf{y}],
\]
where $\mathbf{X}_Q \in \mathbb{R}^{n \times p_Q}$ represents quasi-identifiers observable to an attacker, $\mathbf{X}_U \in \mathbb{R}^{n \times p_U}$ denotes additional non--quasi-identifier attributes that are not available to the attacker, and $\mathbf{y} \in \mathbb{R}^n$ is a confidential target attribute.

The released fully synthetic dataset is
\[
\mathbf{Z}^{(s)} = [\mathbf{X}_Q^{(s)},\; \mathbf{X}_U^{(s)},\; \mathbf{y}^{(s)}],
\]
which is available to the attacker. The attacker observes the real quasi-identifiers $\mathbf{X}_Q$ together with the synthetic data $\mathbf{Z}^{(s)}$, but has no access to $\mathbf{X}_U$ or $\mathbf{y}$ from the original data.

An attacker trains a predictive model $\mathcal{M}$ on the synthetic data
\[
(\mathbf{X}_Q^{(s)}, \mathbf{y}^{(s)})
\]
to obtain parameter estimates $\hat{\Theta}^{(s)}$. This model is then applied to the real quasi-identifiers $\mathbf{X}_Q$ to produce predictions $\hat{\mathbf{y}}$ of the confidential attribute. Inference at attack time is therefore based solely on the observed quasi-identifiers, reflecting a realistic external attacker scenario without access to auxiliary real data.

For notational simplicity, we henceforth write $\mathbf{Z} = [\mathbf{X}_Q,\; \mathbf{y}] = [\mathbf{X},\; \mathbf{y}]$, where $\mathbf{X}$ denotes the quasi-identifiers, and define $\mathbf{Z}^{(s)}$ analogously. This simplification is justified since the unreleased attributes $\mathbf{X}_U$ and their synthetic counterparts $\mathbf{X}_U^{(s)}$ play no role in the proposed risk estimation. 

We assume that the prediction vector $\hat{\mathbf{y}}$ has length $n$, corresponding to the number of records in the original microdata. This assumption is made without loss of generality, as RAPID evaluates risk at the record level conditional on available quasi-identifiers, allowing the number of records in the original data, synthetic data, and attacker's prediction set to differ in practice. Figure \ref{fig:prediction} illustrates the inferential disclosure threat model underlying RAPID, where an adversary uses external knowledge of individuals' quasi-identifiers to train a predictive model on the released synthetic data and infer their sensitive attributes.

\begin{center}
\begin{figure}[!htp]
    \centering
    \includegraphics[width=0.7\linewidth]{flowchart_inferential_risk.pdf}
    \caption{Inference disclosure threat model for synthetic data. A data custodian generates synthetic data from original data containing 
quasi-identifiers ($X$) and a sensitive attribute ($y$). An adversary with access to both the released synthetic data and external information 
sources containing original quasi-identifiers can train a predictive model and infer sensitive attributes of individuals in the original 
dataset.}
    \label{fig:prediction}
\end{figure}
\end{center}

\textcolor{red}{RM: Regarding Figure \ref{fig:prediction}: So far, we have used the terms intruder, adversary, and attacker. It may be preferable to use a single term consistently. Also the notation in the Figure \ref{fig:prediction} is not consistent. The X in external information is not equivalent to the X in original data. The external information is basically the population $\mathcal{X}$ that can contain additional records not present in X.}

\subsection{RAPID for a Categorical Sensitive Attribute}

When \(\mathbf{y}\) is categorical, we define risk in terms of the model-assigned probability to the true class label. For each record \(i\), the synthetic-trained model assigns probability 
\[
g_i = \Pr(\hat{y}_i = y_i \mid \mathbf{x}_i, \hat{\Theta}^{(s)})
\]
to the true class \(y_i\). To evaluate whether this confidence is unusually high, we compare it against a baseline \(b_i\) defined as the marginal proportion of class \(y_i\) in the original data:
\[
b_i = \frac{1}{n}\sum_{j=1}^{n} \mathbb{I}(y_j = y_i)
\]
This baseline represents the prediction confidence achievable by simply guessing according to the marginal class distribution, without using any quasi-identifier information. While an attacker may only observe the synthetic marginals, we use the original marginals for evaluation to reflect a worst-case scenario in which 
the attacker has access to the true marginal distribution through external sources. 
We compute a normalized gain score,
\[
r_i = \frac{g_i - b_i}{1 - b_i}
\]
that measures the improvement in prediction confidence over baseline, normalized by the maximum possible improvement (reaching perfect confidence \(g_i = 1\)). 
The score satisfies:
\begin{itemize}
    \item \(r_i < 0\): model performs worse than baseline 
    \item \(r_i = 0\): model performs at baseline (no information gain)
    \item \(0 < r_i < 1\): partial information gain from quasi-identifiers
    \item \(r_i = 1\): perfect prediction (complete disclosure)
\end{itemize}

% \textcolor{red}{We refer to $r_i$ as the \emph{Relative Confidence Inference Rate (RCIR)}. 
% RCIR is a record-level statistic that quantifies how much more confident the attacker’s prediction for the true class is compared to a baseline strategy that ignores quasi-identifiers.}


A record is considered at risk if \(r_i > \tau\), where \(\tau\) is a policy-defined threshold. \textcolor{red}{Based on empirical validation (Section~\ref{sec:simulations}), 
we recommend \(\tau = 0.3\) as a default, representing 30\% of the maximum possible improvement over baseline.OT schreibt das noch um, dass es mit den Empfehlungen aus den Simulations übereinstimmt} The categorical RAPID metric is then:
\[
\text{RAPID}^{\text{cat}}(\tau) = \frac{1}{n}\sum_{i=1}^{n} \mathbb{I}(r_i > \tau)
\]
representing the proportion of records for which the synthetic data enables inference substantially better than the baseline rate.

\subsection{RAPID for a Continuous Sensitive Attribute}

\textcolor{red}{RM: this prediction error has weird behavior if $y_i = 0$. If the absolute difference is, e.g., 3 or 0.001 we get huge $e_i$: 300000000 and 1000000.}

When the confidential attribute \(\mathbf{y}\) is continuous, we assess whether the model prediction \(\hat{y}_i\) is sufficiently close to the true value \(y_i\). We compute the relative prediction error as a percentage:
\[
e_i = \frac{|y_i - \hat{y}_i|}{|y_i| + \delta} \times 100
\]
where \(\delta > 0\) is a small constant (e.g., \(10^{-6}\)) that prevents division by zero when \(y_i = 0\). A record is at-risk if \(e_i < \varepsilon\), where \(\varepsilon\) is a percentage threshold (typically 5\%--20\%). The continuous RAPID metric is:
\[
\text{RAPID}^{\text{cont}}(\varepsilon) = \frac{1}{n}\sum_{i=1}^{n} \mathbb{I}(e_i < \varepsilon)
\]
representing the proportion of records where predictions fall within the specified error tolerance. For attributes where percentage error is less interpretable, absolute error metrics (e.g., MAE, RMSE) \textcolor{red}{(RM: aber weiter oben wird doch argumentiert, dass mean squared error nicht geeignet sei?)} may be used instead.

\subsection{Computation}

Algorithm~\ref{alg:rapid} formalizes the RAPID evaluation protocol: train a predictive model on synthetic data, apply it to original quasi-identifiers, 
compute per-record risk scores, and aggregate to obtain RAPID.

\begin{algorithm}[H]
\caption{RAPID Evaluation Protocol}
\label{alg:rapid}
\begin{algorithmic}[1]
\Require Original data $\mathbf{Z} = [\mathbf{X}, \mathbf{y}]$; synthetic data $\mathbf{Z}^{(s)}$
\Require Thresholds $\tau \in (0,1)$ (for categorical sensitive attribute, \textcolor{red}{default: $\tau = 0.3$}) and $\varepsilon > 0$ (for continuous sensitive attribute, \textcolor{red}{default: $\varepsilon = 10\%$})

\State Fit predictive model $\mathcal{M}$ on $\mathbf{Z}^{(s)}$ to obtain $\hat{\Theta}^{(s)}$

\State Compute predictions $\hat{\mathbf{y}} = \mathcal{M}(\mathbf{X}; \hat{\Theta}^{(s)})$

\For{$i = 1, \ldots, n$}
    \If{$y$ is categorical}
        \State $g_i \gets \Pr(\hat{y}_i = y_i \mid \mathbf{x}_i)$
        \State $b_i \gets n^{-1} \sum_{j=1}^n \mathbb{I}(y_j = y_i)$
        \State $r_i \gets (g_i - b_i)/(1 - b_i)$
        \State $I_i \gets \mathbb{I}(r_i > \tau)$ 
    \Else
        \State $e_i \gets |y_i - \hat{y}_i|/(|y_i| + \delta)$
        \State $I_i \gets \mathbb{I}(e_i < \varepsilon)$ 
    \EndIf
\EndFor

\State \textbf{Output:}
record-level risks $\{r_i\}$ or $\{e_i\}$,
indicators $\{I_i\}$, and
\[
\text{RAPID} \gets \frac{1}{n} \sum_{i=1}^n I_i
\]
\end{algorithmic}
\end{algorithm}

RAPID is thus an empirical estimate of the probability that a randomly selected record is subject to successful attribute inference under the specified attack model.


\color{blue}

%OLD:

%\begin{algorithm}[H]
%\caption{RAPID Evaluation Protocol}
%\label{algo1}
%\begin{algorithmic}[1]
%\Require Original data $\mathbf{Z} = [\mathbf{X}, \mathbf{y}]$, synthetic data $\mathbf{Z}^{(s)}$

%\State Fit model $\mathcal{M}$ on $\mathbf{Z}^{(s)}$ to obtain $\hat{\Theta}^{(s)}$

%\State Apply $\hat{\Theta}^{(s)}$ to $\mathbf{X}$ for predictions

%\State Compute per-record risk scores:
%\begin{itemize}
 %   \item \textbf{Categorical:} $r_i = \frac{g_i - b_i}{1 - b_i}$ where 
  %  $g_i = \Pr(\hat{y}_i = y_i \mid \mathbf{x}_i)$, 
  %  $b_i = \frac{1}{n}\sum_{j} \mathbb{I}(y_j = y_i)$
  %  \item \textbf{Continuous:} $e_i = \frac{|y_i - \hat{y}_i|}{|y_i| + \delta} \times 100$
%\end{itemize}

%\State Flag at-risk records:
%\begin{itemize}
%    \item \textbf{Categorical:} $r_i > \tau$ (default: $\tau = 0.3$)
%    \item \textbf{Continuous:} $e_i < \varepsilon$ (default: $\varepsilon = 10\%$)
%\end{itemize}

%\State Outputs: Individual risks and flags $\forall i \in \{1, ..., n\}$, $\text{RAPID} = \frac{\text{\# at-risk records}}{n}$

%\end{algorithmic}
%\end{algorithm}

\color{black}

When evaluating multiple models $\mathcal{M} \in \mathcal{S}$, report both 
the average and conservative envelope:
\[
\overline{\text{RAPID}}(\cdot) = \frac{1}{|\mathcal{S}|}\sum_{m\in \mathcal{S}}\text{RAPID}^{(m)}(\cdot), 
\qquad 
\text{RAPID}_{\max}(\cdot) = \max_{m\in \mathcal{S}}\text{RAPID}^{(m)}(\cdot)
\]

\subsection{Implementation Considerations}

RAPID is applicable to any algorithm producing class probabilities or point predictions. To reflect a strong attacker, we recommend 
evaluating powerful learners (e.g., random forests, gradient boosting) and reporting the maximum observed risk. When multiple synthetic datasets are 
available, risk can be averaged across them or reported as the worst-case value, depending on desired conservativeness. Thresholds \(\tau\) and \(\varepsilon\) 
should reflect policy requirements; we recommend defaults of \(\tau = 0.3\) and \(\varepsilon = 10\%\), \textcolor{red}{noch to be validated in Section~\ref{sec:simulations}}. For data-driven selection, one can permute \(\mathbf{y}\), recompute RAPID, and choose the threshold placing observed risk above the 95th percentile of the 
resulting null distribution. Uncertainty can be quantified via bootstrap resampling or, treating RAPID as a binomial proportion, using Wilson score or Clopper–Pearson confidence intervals. \textcolor{red}{missing sources suchen}

\paragraph{Evaluating Synthetic Data Generators.}

To assess the general disclosure risk of a synthetic data generator (SDG), we recommend $k$-fold cross-validation: partition the original data into $k$ folds (stratified by $\mathbf{y}$ if categorical), generate synthetic data from $k$-$1$ folds, and evaluate RAPID on the held-out fold. Aggregating across folds yields the expected risk $\mathbb{E}$[RAPID] with confidence intervals, providing a robust assessment of the SDG's average disclosure risk. This approach is implemented via the \texttt{rapid\_synthesizer\_cv()} function and is particularly useful for selecting between alternative synthesis methods (e.g., CART vs. parametric) or optimizing hyperparameters. Once a method is selected, standard RAPID should be applied to the final synthetic data product to obtain record-level risk assessments prior to the data release.



\subsection{Properties}

RAPID has several structural properties that follow directly from its definition.
First, it targets successful and confident attribute inference events rather than aggregate predictive accuracy, aligning the risk measure with inferential disclosure in the classical SDC taxonomy. Second, the normalization by the empirical base rate ensures that categorical risk scores are invariant to class imbalance, so that common outcomes do not spuriously inflate disclosure risk.

Third, RAPID is bounded in $[0,1]$, as it is defined as the empirical mean of record-level disclosure indicators. This facilitates comparison across datasets, synthesizers, and attacker models. Fourth, the construction is model-agnostic \textcolor{red}{("model-agnostic" has two meanings in this paper: a) you use a model, but it doesn't matter which one, b) you don't use a model at all - see matching-based disclosure risk measures)}: any predictive model capable of producing point predictions or class probabilities may be used, allowing RAPID to be evaluated under strong and adaptive attacker assumptions.

Finally, RAPID is monotone in the decision thresholds $\tau$ and $\varepsilon$, respectively. Increasing the categorical threshold $\tau$ or decreasing the continuous tolerance $\varepsilon$ can only reduce the number of records classified as at risk. This monotonicity supports transparent sensitivity analysis and makes RAPID well suited for practical disclosure control.

Because RAPID operates at the record level, it also enables subgroup-specific and conditional risk analysis by restricting the indicator average to subsets of the quasi-identifier space (see, e.g., Figure \ref{fig:sim3}).


% RAPID offers several advantages: (1) it explicitly targets confidently correct inferences rather than general accuracy, (2) baseline normalization ensures 
% robustness to class imbalance, (3) the metric lies between 0 and 1 as a proportion, enabling comparison across datasets and methods, (4) model-agnostic 
% design allows evaluation against strong attackers, and (5) it responds monotonically to threshold changes, making interpretation straightforward for 
% disclosure control practice.



\section{Toy Example}

We demonstrate the categorical RAPID metric on a simple example. Consider a dataset with 100 records where class \texttt{healthy} has 60\% prevalence 
(marginal frequency \textcolor{red}{(better "proportion" than "frequency"? Because we divide through n)}). A model trained on synthetic data is applied to three original records, all truly belonging to class \texttt{healthy}:

\begin{itemize}
    \item Record 1: $g_1 = \Pr(\hat{y}_1 = \text{healthy} \mid \mathbf{x}_1) = 0.70$
    \item Record 2: $g_2 = \Pr(\hat{y}_2 = \text{healthy} \mid \mathbf{x}_2) = 0.85$
    \item Record 3: $g_3 = \Pr(\hat{y}_3 = \text{healthy} \mid \mathbf{x}_3) = 0.55$
\end{itemize}

The baseline is the marginal frequency of class \texttt{healthy} in the original data:
\[
b_i = 0.60 \quad \text{for all three records (same class)}
\]

We compute the normalized gain for each record:
\begin{align*}
r_1 &= \frac{0.70 - 0.60}{1 - 0.60} = \frac{0.10}{0.40} = 0.25 \\
r_2 &= \frac{0.85 - 0.60}{1 - 0.60} = \frac{0.25}{0.40} = 0.625 \\
r_3 &= \frac{0.55 - 0.60}{1 - 0.60} = \frac{-0.05}{0.40} = -0.125
\end{align*}

\textbf{Interpretation:}
\begin{itemize}
    \item Record 1: 25\% of maximum improvement over baseline
    \item Record 2: 62.5\% of maximum improvement (high confidence!)
    \item Record 3: Below baseline (model worse than guessing)
\end{itemize}

Using the \textcolor{red}{recommended threshold $\tau = 0.3$}, we flag records where $r_i > 0.3$:
\[
\mathbb{I}\{r_1 > 0.3\} = 0, \quad
\mathbb{I}\{r_2 > 0.3\} = 1, \quad
\mathbb{I}\{r_3 > 0.3\} = 0
\]

The RAPID metric is:
\[
\text{RAPID}^{\text{cat}}(0.3) = \frac{1}{3}(0 + 1 + 0) = 0.33
\]

This indicates that 33\% of records have predictions substantially better than the baseline rate. Record 2, with 62.5\% normalized gain, represents a high 
disclosure risk: the synthetic data enables confident inference beyond what marginal class frequencies alone would allow. This example illustrates how 
RAPID identifies records where quasi-identifiers provide meaningful information gain, rather than merely measuring prediction accuracy.

\vspace{1em}
\noindent
\textbf{Example (continuous attribute):} 

Now consider the case where the sensitive variable is continuous, such as income. Suppose an attacker trains a regression model on synthetic data and applies it 
to the original covariates of three individuals. Let the true incomes and the model's predictions be:

\begin{itemize}
    \item Record 1: $y_1 = 50{,}000$, $\hat{y}_1 = 47{,}000$
    \item Record 2: $y_2 = 35{,}000$, $\hat{y}_2 = 39{,}000$
    \item Record 3: $y_3 = 80{,}000$, $\hat{y}_3 = 90{,}000$
\end{itemize}

The relative prediction errors (as percentages of true values) are:
\begin{align*}
e_1 &= \frac{|50{,}000 - 47{,}000|}{50{,}000} \times 100 = 6\% \\
e_2 &= \frac{|35{,}000 - 39{,}000|}{35{,}000} \times 100 = 11.4\% \\
e_3 &= \frac{|80{,}000 - 90{,}000|}{80{,}000} \times 100 = 12.5\%
\end{align*}
Using a threshold of $\varepsilon = 10\%$, we flag records where predictions 
fall within 10\% of the true value:
\[
\mathbb{I}\{e_1 < 10\%\} = 1, \quad
\mathbb{I}\{e_2 < 10\%\} = 0, \quad
\mathbb{I}\{e_3 < 10\%\} = 0
\]

The continuous RAPID metric is:
\[
\text{RAPID}^{\text{cont}}(10\%) = \frac{1}{3}(1 + 0 + 0) = 0.33
\]

This indicates that for one-third of individuals, the synthetic-trained model predicted income within 10\% relative error. Record 1, with only 6\% error, represents a disclosure risk: the synthetic data enables an attacker to infer income with high precision. This form of attribute disclosure is directly relevant for risk assessment but would not be captured by aggregate metrics like mean absolute error alone.



\subsection{Software and defaults}\label{subsec:software}

The proposed risk measure is implemented in \textsf{R}.
% , \textsf{Python}, and \textsf{Julia}, ensuring accessibility across the most common data science environments. 
A dedicated package, \texttt{riskutility}, accompanies this paper and is already publicly available at \textcolor{red}{CRAN / GitHub XXX}.

\color{red}

In \textsf{R}, the package is available on the Comprehensive R Archive Network (CRAN) and can be installed via \texttt{install.packages("riskutility")}. The reference implementation builds on well-established libraries: \texttt{synthpop} for synthetic data generation, \texttt{ranger} for random forests, \texttt{xgboost} for gradient boosting, and \texttt{glmnet} for regularized generalized linear models. Additional utilities for resampling and inference are supported by \texttt{caret} and \texttt{boot}.

In \textsf{Python}, the package is distributed via the Python Package Index (PyPI) and can be installed using \texttt{pip install riskutility}. It integrates with \texttt{scikit-learn} for model training and prediction, and supports synthetic data generation using \texttt{ctgan} or \texttt{SDV}. Common scientific computing packages such as \texttt{pandas}, \texttt{numpy}, and \texttt{scipy} are used for data manipulation and evaluation.

In \textsf{Julia}, the package is registered in the General Julia Registry and can be added via Julia's package manager with \texttt{import Pkg; Pkg.add("RiskUtility")}. The Julia implementation uses \texttt{MLJ.jl} for supervised learning models and provides native support for tabular data and statistical computation using the Julia ecosystem.

All three versions provide consistent functionality for computing the RAPID metric for both categorical and continuous confidential attributes. This includes attacker models, baseline normalization, threshold calibration, and bootstrap-based uncertainty quantification. The unified interface ensures comparability across synthesizers, attacker setups, and platforms.

\color{black}

Unless otherwise noted, we adopt the following default settings in our experiments: the attacker model \(\mathcal{M}\) is a random forest with 500 trees and probabilistic outputs enabled. \textcolor{red}{For categorical attributes, we evaluate RAPID at threshold values \(\tau \in \{1.1, 1.25, 1.5\}\); for continuous attributes, we consider relative error tolerances \(\varepsilon \in \{0.05, 0.10, 0.20\}\). Uncertainty is quantified via a nonparametric bootstrap over the original dataset (500 replicates), with percentile-based confidence intervals.}

% \subsection{End-to-end protocol}\label{subsec:protocol}
% Given original microdata \(\mathbf{Z}=[\mathbf{X},\mathbf{y}]\), choose a synthesizer family and (optionally) generate \(M\) synthetic replicates \(\{\mathbf{Z}^{(s,m)}\}_{m=1}^M\). For each synthetic replicate:
% \begin{enumerate}
% \item \textbf{Train attacker on synthetic:} fit \(\hat{\Theta}^{(s)} \leftarrow \mathcal{M}\!\big(\mathbf{y}^{(s)}\sim \mathbf{X}^{(s)}\big)\).
% \item \textbf{Predict on real covariates:} obtain \(\hat{\mathbf{y}}=f(\mathbf{X};\hat{\Theta}^{(s)})\).
% \item \textbf{Compute per-record scores:}
%   \begin{itemize}
%   \item Categorical \(y\): \(g_i = \Pr(\hat{y}_i=y_i\mid \mathbf{x}_i,\hat{\Theta}^{(s)})\); baseline \(b_i=\tilde{p}_{y_i}\) as the average of \(g_j\) over all \(j\) with \(y_j=y_i\); relative score \(r_i=g_i/b_i\).
%   \item Continuous \(y\): \(h_i=\mathbb{I}\!\left(\left|\frac{y_i-\hat{y}_i}{\hat{y}_i}\right|<\varepsilon\right)\) (or use \(|y_i-\hat{y}_i|/(\lvert y_i\rvert+\lvert \hat{y}_i\rvert)\) when needed).
%   \end{itemize}
% \item \textbf{Aggregate to risk:} 
% \(\mathrm{RCIR}^{\text{cat}}(\tau)=\frac{1}{n}\sum_i \mathbb{I}\{r_i>\tau\}\) or \(\mathrm{RCIR}^{\text{cont}}(\varepsilon)=\frac{1}{n}\sum_i h_i\).
% \end{enumerate}
% If multiple strong learners are considered (\(\mathcal{S}\)), report both the average and a conservative envelope:
% \[
% \overline{\mathrm{RCIR}}(\cdot)=\frac{1}{|\mathcal{S}|}\sum_{m\in \mathcal{S}}\mathrm{RCIR}^{(m)}(\cdot), 
% \qquad 
% \mathrm{RCIR}_{\max}(\cdot)=\max_{m\in \mathcal{S}}\mathrm{RCIR}^{(m)}(\cdot).
% \]
% To calibrate thresholds, we recommend a permutation null: permute \(\mathbf{y}\) in \(\mathbf{Z}\) \(B\) times, recompute \(\mathrm{RCIR}\) under each permutation, and choose the smallest \(\tau\) (or largest \(\varepsilon\)) such that the observed \(\mathrm{RCIR}\) lies below the 95th percentile of the null distribution.

\section{Real data illustration: UCI Adult (Census Income) WIP}\label{sec:adult}
We illustrate the workflow on the \emph{Adult} dataset (UCI Machine Learning Repository; 48{,}842 rows, 14 attributes; binary confidential attribute \(\,\mathbf{y}=\) ``income\(\,>\)\$50K'') \citep{adult_2}. The covariates, $\mathbf{X}$, include age, education, hours-per-week, marital status, etc. We treat \(\mathbf{y}\) as confidential, sensitive variable and \(\mathbf{X}\) as potentially known quasi-identifiers.

\paragraph{Pre-processing.}
We standardize continuous features within the original file (means/SDs computed on \(\mathbf{Z}\) and then applied to \(\mathbf{Z}^{(s)}\) to avoid leakage), and one-hot encode categorical predictors consistently across original and synthetic files.

\paragraph{Synthesizers.}
We consider three representative mechanisms: (i) a CART-based tabular synthesizer (as implemented in \texttt{synthpop} \citep{raab2024synthpop}); (ii) a parametric GLM with Gaussian/Multinomial conditionals \textcolor{red}{REFERENCE}; (iii) a neural tabular synthesizer (using CTGAN \textcolor{red}{REFERENCE}). Each synthesizer is trained solely on \(\mathbf{Z}\) and produces \(M=5\) synthetic replicates.

\paragraph{Attack models.}
We evaluate an attacker suite \(\mathcal{S}=\{\)RF, GBM, \(\ell_1\)-logistic\(\}\), trained on each synthetic replicate and scored on the real covariates \(\mathbf{X}\).

\paragraph{Metrics and reporting.}
For \(\tau \in \{0.3,0.4,0.5\}\) we report \(\mathrm{RAPID}^{\text{cat}}(\tau)\) along with 95\% bootstrap CIs (500 resamples of \(\mathbf{Z}\)); for each synthesizer we show the simple average across \(M\) replicates and the worst-case envelope. We additionally stratify \(\mathrm{RAPID}\) by true class \(y\) to reveal asymmetric risks. \textcolor{red}{the last sentence needs explanation}


\begin{table}[!h]
\centering
\caption{\label{tab:rcir-adult}RAPID-based Relative Confidence Inference Rate by attacker model and threshold for Adult income category (categorical confidential attribute). Values are proportions; 95\% bootstrap CIs in parentheses. \textcolor{red}{Achtung: jetzt ist unser default tau 0.3 nicht verwendet worden. Code is already there for extension to Synthesizer ranger and additional attacker with CART, but synthpop with ranger is very slow. Code for the max envolope and new table is also already there, but calculations are still running.}}
\centering
\resizebox{\ifdim\width>\linewidth\linewidth\else\width\fi}{!}{
\begin{tabular}[t]{llccc}
\toprule
Synthesizer & Attacker & $\tau = 1.10$ & $\tau = 1.25$ & $\tau = 1.50$\\
\midrule
CART & RF & 0.4832 (0.4775–0.4874) & 0.0908 (0.0883–0.0920) & 0.0510 (0.0490–0.0529)\\
CART & GBM & 0.4921 (0.4840–0.4968) & 0.0922 (0.0893–0.0933) & 0.0612 (0.0600–0.0628)\\
CART & LOGIT & 0.4931 (0.4897–0.4964) & 0.0850 (0.0823–0.0865) & 0.0465 (0.0448–0.0483)\\
\bottomrule
\end{tabular}}
\end{table}



\paragraph{Sensitivity and diagnostics.}
To better understand model behavior and validate our results, we conduct the following diagnostic checks:
\begin{enumerate}
\item \textbf{Threshold curves:} We plot $\mathrm{RCIR}^{\text{cat}}(\tau)$ across a grid of $\tau$ values to visualize how disclosure risk decays as stricter confidence thresholds are imposed.
\item \textbf{Class balance:} For each class $k$, we report the baseline confidence $\tilde{p}_k$ to expose the influence of class prevalence on the relative risk.
\item \textbf{Calibration:} We generate reliability diagrams for $g_i$ (the model-predicted probability for the true class) based on synthetic-trained models scored on real covariates. If calibration issues are detected, we apply Platt or isotonic recalibration using a held-out portion of $\mathbf{Z}$. \textcolor{red}{this needs further explanation}
\item \textbf{Joint utility–risk view:} To contextualize disclosure risks, we also report utility metrics such as the predictive accuracy or AUC of models trained on $\mathbf{Z}^{(s)}$ but evaluated on true labels in $\mathbf{Z}$, along with marginal and low-order moment comparisons. \textcolor{red}{Check: haben wir AUC etc wirklich gemacht?}
\end{enumerate}


To assess inferential disclosure risk in synthetic data, we applied the RAPID (Relative Attribute Prediction–Induced Disclosure) metric to the UCI Adult dataset, using \texttt{income} as the sensitive attribute. Five synthetic datasets were generated via the CART synthesizer (\texttt{synthpop} package). For each replicate, we trained a random forest model on the synthetic data to infer the sensitive attribute from a set of quasi-identifiers. The model’s predictive performance was evaluated on the original dataset.

The RAPID metric measures how confidently a model trained on synthetic data can correctly predict sensitive values in the original data. \textcolor{red}{RCIR sollte $\bar{r}$ sein, right?} Specifically, we compute the \emph{Relative Confidence Inference Rate (RCIR)}, defined as the proportion of records for which the predicted probability for the true class is unusually high compared to average predictions for that class. A high RCIR would indicate that an attacker could infer the true value of a sensitive attribute with unjustifiably high confidence, posing a disclosure risk. 

We set the relative confidence threshold at $\tau = 1.25$ \textcolor{red}{Achtung: das ist nicht der aktuelle Defaultwert}, meaning that we count as risky only those cases where the model's predicted probability for the true class is at least 25\% higher than its baseline (i.e., class-conditional average). This value strikes a balance between sensitivity to confident predictions and tolerance for classification uncertainty, reflecting a moderate privacy concern.

To quantify uncertainty around RCIR estimates, we performed non-parametric bootstrapping with $R = 500$ resamples from the original dataset. This approach accounts for sample variability and provides robust percentile-based confidence intervals, enabling a more reliable interpretation of disclosure risk.

Across the five synthetic replicates, the RCIR values were consistently low (see Table~\ref{tab:rcir}), ranging from 0.0896 to 0.0920. All associated 95\% confidence intervals remained narrow, with upper bounds below 9.3\%, suggesting a low inferential disclosure risk under the given attack model and threshold.

\begin{table}[!htp]
\centering
\caption{RAPID-based Relative Confidence Inference Rate (RCIR) estimates ($\tau = 1.25$) and bootstrapped 95\% confidence intervals for each synthetic replicate.}
\label{tab:rcir}
\renewcommand{\arraystretch}{1.3}
\begin{tabular}{>{$}l<{$} c @{\,} c @{\,} c}
\toprule
\text{Replicate} & \text{RCIR} & \text{95\% CI (Lower)} & \text{95\% CI (Upper)} \\
\midrule
1 & 0.0911 & 0.0879 & 0.0919 \\
2 & 0.0906 & 0.0877 & 0.0922 \\
3 & 0.0896 & 0.0877 & 0.0907 \\
4 & 0.0907 & 0.0885 & 0.0918 \\
5 & 0.0920 & 0.0897 & 0.0928 \\
\bottomrule
\end{tabular}
\end{table}


To better understand how the relative inference confidence threshold $\tau$ affects the disclosure risk, we computed the Relative Confidence Inference Rate (RCIR) across a range of $\tau$ values from 1.0 to 2.0 in increments of 0.05. Figure~\ref{fig:rcir-curve} illustrates how RCIR drops steeply as $\tau$ increases, demonstrating the sensitivity of the metric to changes in the attacker’s confidence level.

At $\tau = 1.0$, which reflects a naive attacker accepting any prediction stronger than the baseline average, approximately 64\% of records are considered high-risk. As the threshold increases, RCIR rapidly declines; by $\tau = 1.25$, only 9.1\% of records are flagged as risky. This indicates that only a small fraction of predictions reach a level of relative confidence that may be deemed problematic under moderate privacy expectations. At $\tau = 1.65$ and beyond, the RCIR reaches zero—implying that the attacker can no longer make any highly confident inferences under this stricter standard.

This threshold curve helps calibrate the RAPID metric by allowing analysts to balance privacy sensitivity and acceptable inference risk. Choosing $\tau$ based on such empirical curves enables data stewards to define meaningful disclosure thresholds grounded in observed model behavior.

\begin{figure}[!htp]
\centering
\includegraphics[width=0.75\textwidth]{rcir_curve_quantile.pdf}
\caption{RCIR as a function of the confidence threshold $\tau$ for the UCI Adult dataset. The curve demonstrates a sharp decline in risky inferences as the threshold increases, reaching near-zero risk for $\tau > 1.6$.}
\label{fig:rcir-curve}
\end{figure}

To further explore how the inferential disclosure risk responds to varying attacker confidence levels, we examined the relationship between the RCIR metric and the confidence threshold parameter~$\tau$ across five synthetic datasets (Figure~\ref{fig:rcir_curve}). For each threshold value in the range $[1.0, 2.0]$, we computed the mean RCIR and the confidence interval, i.e. its empirical 5th and 95th percentiles across replicates.

The results show a sharp drop in inferential disclosure risk between $\tau = 1.10$ and $\tau = 1.15$, where the RCIR plummets from approximately $49\%$ to $11\%$. Beyond this point, RCIR decreases more gradually until about $\tau = 1.60$, after which it stabilizes at zero, indicating that no confident inferences remain under stricter attacker thresholds.

The extremely narrow quantile bands, with widths typically below 1--2 percentage points, demonstrate the robustness of the RAPID metric across synthetic replicates. This confirms that inferential risk estimates are stable even under the stochastic nature of data synthesis.

These results underscore the importance of carefully selecting the confidence threshold~$\tau$ to align with acceptable levels of risk. For instance, setting $\tau = 1.25$ leads to an RCIR of approximately $9\%$, suggesting that under this threshold, only a small fraction of synthetic records could lead to confident and correct inferences about the sensitive attribute. The flat region beyond $\tau = 1.65$ suggests diminishing utility for attackers, reinforcing the interpretability and practical usability of the $\tau$ parameter in disclosure risk control.

\section{Simulation study}
\label{sec:simulations}
\subsection{Overview and Design}

To validate RAPID and investigate factors influencing attribute-inference 
risk (per row), we conducted four simulation studies:
\begin{enumerate}
\item \textbf{Dependency strength:} How does risk scale with QI--sensitive 
attribute relationships? ($\kappa \in [0, 100]$)

\item \textbf{Threshold sensitivity:} How does $\tau$ affect risk across 
dependency regimes? (5 $\kappa$ levels × 19 $\tau$ values)

\item \textbf{QI attribution:} Which quasi-identifiers drive risk? 
(Regression-based analysis)

\item \textbf{Attacker robustness:} Is RAPID consistent across models? 
(RF, CART, GBM comparison)
\end{enumerate}

All simulations use synthetic health microdata with six variables (gender, age, education, income, health score, disease status). We control dependency strength via a global parameter $\kappa \geq 0$, with full details in 
Appendix~\ref{app:datagen}.

\subsection{Data Generation Process}

We simulate $n$ independent records with six variables: gender ($G$), age ($A$), education ($E$), income ($I$), health score ($H$), and disease status ($D$). The design encodes realistic dependencies through a latent socioeconomic 
status (SES) variable, with dependency strength controlled by parameter $\kappa \geq 0$. 

Signal and noise weights are derived from $\kappa$ as:
\[
w_{\text{signal}} = \sqrt{\frac{\kappa}{1+\kappa}}, \quad 
w_{\text{noise}} = \sqrt{\frac{1}{1+\kappa}}.
\]

At $\kappa=0$, relationships are weak ($w_{\text{signal}} \approx 0$); at $\kappa \gg 1$, dependencies approach deterministic strength ($w_{\text{signal}} \to 1$). Age follows a truncated normal; education is ordinal derived from a latent variable; income is log-linear; health score uses sigmoid transformation; disease status is generated via 
multinomial logit with $\kappa$-scaled coefficients; gender is binary with mild SES dependency. Complete mathematical specifications appear in Appendix~\ref{app:datagen}.

\subsection{Dependency Strength}
We varied dependency parameter $\kappa$ from 0 to 100 (101 values, 10 replications each) to investigate how attribute-inference risk scales with quasi-identifier--sensitive attribute relationship strength. Subplot a) in Figure \ref{fig:sim1_2} shows a S-shaped trajectory demonstrates that risk escalates rapidly when transitioning from weak to moderate dependencies, then saturates as relationships approach deterministic strength. Saturation near 0.97 rather than 1.0 reflects inherent noise from the CART synthesizer's sampling. Attacker accuracy follows a similar trajectory (0.70 to 0.98), demonstrating that RAPID reliably reflects actual prediction performance: datasets with high RAPID face genuinely elevated disclosure risk.

Having established that risk scales with dependency strength, we next investigate how the threshold parameter $\tau$ interacts with this dependency structure. Simulation 2 addresses this challenge by examining threshold sensitivity across the full dependency spectrum.

\subsection{Threshold Sensitivity}
We examined how confidence threshold $\tau$ affects RAPID across five dependency levels ($\kappa \in \{0, 5, 10, 20, 50\}$), varying $\tau$ from 0.05 to 0.95 in 0.05 increments (10 replications per combination). Subplot b) in Figure \ref{fig:sim1_2} reveals a qualitative shift in curve geometry: at low dependency ($\kappa=0$, gray), RAPID decreases convexly, dropping rapidly from 0.50 to near-zero by $\tau=0.60$, indicating diffuse attacker confidence. At high dependency ($\kappa \geq 5$, blue/red), curves become concave, remaining elevated 
($>0.85$) until stringent thresholds ($\tau > 0.70$), reflecting concentrated confidence distributions near certainty. This convex-to-concave transition marks a shift from weak to strong adversarial inference capability. Practically, data curators should choose higher thresholds ($\tau \geq 0.70$) for strongly dependent data to avoid flagging nearly all records, while lower thresholds ($\tau \approx 0.30$--0.40) suffice for weakly dependent data 
where they effectively separate high-confidence from baseline predictions.

\begin{figure}[!htp]
    \centering
    \includegraphics[width=1\linewidth]{plot_sim1_and_sim2_combined.pdf}
    \caption{Impact of dependency strength and confidence threshold on RAPID:\\
    \\
(a) RAPID and attacker accuracy increase monotonically with dependency 
strength $\kappa$. RAPID rises from 0.25 at $\kappa=0$ to 0.97 at $\kappa=100$, 
with steepest increases at low $\kappa$ values. This 
S-shaped growth demonstrates that attribute-inference risk escalates rapidly 
when transitioning from weak to moderate quasi-identifier--sensitive attribute 
relationships, then saturates as dependencies approach deterministic levels. \\
(b) RAPID vs confidence threshold $\tau$ for varying $\kappa$. At low 
dependency ($\kappa=0$, gray), the curve is convex, reflecting diffuse 
attacker confidence where most records are filtered out at moderate thresholds. 
At high dependency ($\kappa \geq 5$, blue/red), curves become concave, remaining 
elevated until stringent thresholds ($\tau > 0.7$) are applied. This transition 
reflects a qualitative shift in attacker confidence distributions as dependencies 
strengthen.\\
Both panels: Mean $\pm$ 1 SD over 10 simulations; $n=1000$ records, 
CART synthesizer, Random Forest attacker. Panel (a): $\tau=0.3$.
}
    \label{fig:sim1_2}
\end{figure}

\subsection{Quasi-Identifier Attribution}

To identify which quasi-identifiers drive attribute-inference risk, we fitted logistic regression models predicting at-risk status from demographic characteristics across 50 simulations ($\kappa=10$, $\tau=0.3$). 
Figure~\ref{fig:sim3} displays coefficient distributions relative to reference categories (marked in red). This regression-based approach demonstrates how RAPID's per-record risk flags enable granular attribution analysis: by modeling what predicts at-risk status, data curators can identify which quasi-identifier combinations elevate disclosure vulnerability. The analysis reveals heterogeneity across quasi-identifier levels. This differentiation is valuable for understanding risk drivers in synthetic data and assessing which demographic characteristics the attacker exploits most effectively.


\begin{figure}[!htp]
    \centering
    \includegraphics[width=1\linewidth]{plot_sim3_coefficients_minimal.pdf}
    \caption{Attribution analysis of re-identification risk by quasi-identifier. Boxplots display regression coefficients ($\beta$) from logistic models predicting 
at-risk status across 50 simulations ($\kappa$=10, $\tau$=0.3, n=1000). White diamonds = means; red points = reference categories ($\beta$=0). Positive $\beta$ indicates elevated re-identification risk.}
    \label{fig:sim3}
\end{figure}


\subsection{Attacker Model Robustness}

To assess whether RAPID is sensitive to attacker model choice, we compared three tree-based models (Random Forest, CART, GBM) at $\kappa=10$ and $\tau=0.3$ across 50 replications. Table~\ref{tab:sim4} shows that all three models achieve similar RAPID values within a narrow range (0.864--0.876). This robustness is a desirable property for practical risk assessment: RAPID provides consistent risk estimates regardless of which specific tree-based method models the attacker, reducing sensitivity to modeling assumptions about adversarial capabilities. The narrow variation demonstrates that RAPID captures a stable notion of attribute-inference vulnerability rather than idiosyncrasies of particular algorithms. Data curators can thus apply RAPID confidently without requiring precise knowledge of adversary methods, as the metric reliably quantifies disclosure risk across reasonable attacker specifications.


\begin{table}[!htp]
\centering
\caption{Attacker Model Comparison}
\label{tab:sim4}
\renewcommand{\arraystretch}{1.3}
\begin{tabular}{lcccc}
\toprule
Attacker & RAPID & SD & Accuracy & SD \\
\midrule
CART & 0.876 & 0.013 & 0.879 & 0.011 \\
GBM & 0.875 & 0.011 & 0.881 & 0.011 \\
Random Forest & 0.864 & 0.010 & 0.891 & 0.010 \\
\bottomrule
\end{tabular}
\end{table}



\subsection{Computational cost}\label{subsec:cost}
Let \(T_{\text{train}}\) be the time to fit \(\mathcal{M}\) on \(\mathbf{Z}^{(s)}\) and \(T_{\text{score}}\) the time to score \(\mathbf{X}\). Per replicate, the dominant cost is \(T_{\text{train}}\). Overall complexity is \(O\!\big(M\cdot |\mathcal{S}|\cdot T_{\text{train}}\big)\). Bootstrap intervals add a multiplicative factor of \(B\) light-weight recomputations (reusing \(\hat{\Theta}^{(s)}\), only resampling rows of \(\mathbf{Z}\)), so walltime scales roughly as \(O\!\big(M\cdot |\mathcal{S}|\cdot T_{\text{train}} + B\cdot n\big)\).

\subsection{Reproducible \textsf{R} sketch}\label{subsec:r-code}
Below we give a minimal \textsf{R} sketch for the categorical case; continuous \(y\) differs only in the score definition.

RAPID is implemented as an open-source R package (\texttt{rapidsynthpop}) with 
straightforward workflows integrating with the widely-used \texttt{synthpop} 
package for generating synthetic data. A typical evaluation requires only original 
data, synthetic data, and specification of quasi-identifiers and the sensitive 
attribute:

\begin{small}
\begin{verbatim}
library(rapidsynthpop)
rapid_result <- rapid(
  original_data = data_orig,
  synthetic_data = data_syn,
  quasi_identifiers = c("age", "education", "gender"),
  sensitive_attribute = "disease_status",
  model_type = "rf",
  cat_tau = 0.3
)
\end{verbatim}
\end{small}

The function returns sample-level risk rates and per-record risk scores. For categorical sensitive attributes, the primary output is the confidence rate: the proportion of records for which the attacker's predicted probability (relative to the model-estimated baseline) exceeds threshold $\tau$. For continuous attributes, RAPID reports the roportion of records where relative prediction error falls below a specified tolerance.


\section{Discussion}

\textbf{WIP}
\color{red}

Table-based measures like DiSCO \citep{Raab_2025_Practical} assess attribute disclosure by flagging records as disclosive when synthetic data imply a unique target value for a given quasi-identifier combination $q$—that is, when the synthetic column proportion $p_{stq}=1$ within that $q$-group. DiSCO then counts the number of original records in those $q$-groups with that target $t$ (``Correct Original''). Formally,
$$
\text{DiSCO} = 100 \times \sum_{q}\sum_{t} \frac{d_{tq} \mid p_{stq}=1}{N_d},
$$
where $d_{tq}$ denotes original records in quasi-identifier group $q$ with target $t$, and $N_d$ is the total number of original records. For example, if every synthetic record for ``60-year-old male smoker from region A'' shows depression ($p_{\text{depression}|q}=1$), and five original records match these quasi-identifiers with depression, those five records contribute to DiSCO. Related measures include DCAP and TCAP, which use different denominators to express conditional disclosure probabilities.

This approach offers transparency and integrates naturally with classical Statistical Disclosure Control concepts of key variables and within-group homogeneity. However, it has several limitations for modern synthetic data assessment. 

First, DiSCO requires discretizing continuous sensitive attributes and quasi-identifiers into categorical bins to construct contingency tables, making results sensitive to these binning choices. Coarse bins (e.g., three broad age groups) yield larger equivalence classes (i.e., groups of records sharing the same quasi-identifier combination) that are less likely to exhibit perfect homogeneity ($p_{stq}=1$), potentially underestimating risk. Fine bins (e.g., five-year age intervals) create smaller, more homogeneous equivalence classes but also sparser ones with fewer observations. There is no objective criterion for optimal binning, introducing subjectivity into risk assessment.

Second, DiSCO's deterministic threshold ($p_{stq}=1$) is strict: it only flags risk when synthetic data exhibit perfect within-class homogeneity for the sensitive attribute. This misses inferential vulnerabilities from strong but non-deterministic patterns that adversaries using machine learning could exploit. Consider an equivalence class where 85\% of synthetic records show depression and 15\% show anxiety. An adversary training a predictive model on the synthetic data would learn to predict depression for this quasi-identifier pattern with high 
confidence (0.85), as the model captures the strong probabilistic association. This poses genuine disclosure risk for original records with these quasi-identifiers who have depression. Yet because $p_{stq} \neq 1$, DiSCO does not flag these records as at-risk.
More generally, flexible models can exploit complex interactions that don't yield perfect homogeneity in individual equivalence classes. RAPID captures these vulnerabilities by training flexible models that mirror realistic adversarial capabilities and flagging records where model confidence substantially exceeds baseline, regardless of whether synthetic data exhibit $p_{stq}=1$.

Third, while DiSCO identifies which original records are at-risk, it provides only binary classification (disclosive or not) without quantifying the strength of adversarial inference. All records with $p_{stq}=1$ are treated identically regardless of the broader context: whether the quasi-identifier pattern is common or rare in the data, whether the sensitive attribute has high or low marginal prevalence, or whether alternative quasi-identifier combinations would yield similar predictions. RAPID produces individual-level confidence scores that vary continuously, enabling risk stratification: curators can identify not just which records are at elevated risk, but which face the highest vulnerability, facilitating targeted disclosure control and subgroup-specific diagnostics.

Fourth, DiSCO can flag records as at-risk due to synthetic data artifacts rather than genuine inferential vulnerability. If a synthesizer happens to produce a degenerate cell ($p_{stq}=1$) for a quasi-identifier combination through sampling variability or model idiosyncrasies—even when the original data for that combination are heterogeneous—DiSCO counts all matching original records as disclosive. This may overstate risk for combinations where no real adversary could reliably infer 
the sensitive attribute. RAPID avoids this by requiring that a predictive model trained on the synthetic data actually achieves high-confidence correct predictions on the original quasi-identifiers. Risk is only flagged when the attacker's model, using the released synthetic file, demonstrates genuine predictive capability—more closely aligned to a successful inference attack than to synthetic table quirks.

Fifth, DiSCO does not calibrate for class imbalance or baseline prediction difficulty. In datasets with skewed sensitive attribute distributions (e.g., 90\% no condition, 8\% depression, 2\% anxiety), simply predicting the majority class yields high accuracy without genuine inference. DiSCO treats all $p_{stq}=1$ cells equally regardless of whether the unanimous value is common or rare. RAPID addresses this through class-normalized confidence scoring: for categorical targets, it compares the attacker's predicted probability for the true class against a model-estimated marginal baseline. This ensures that risk reflects predictive advantage beyond naive baseline strategies, analogous to \texttt{synthpop}'s baseline scaling for utility metrics but applied to disclosure risk. Records are 
only flagged when the attacker is substantially more confident than chance, calibrating risk assessment to class prevalence.

Sixth, as quasi-identifier dimensionality or cardinality increases, DiSCO's contingency table approach becomes computationally and interpretatively unwieldy. With many high-cardinality categorical variables (e.g., occupation codes, geographic units), tables explode combinatorially and most cells become sparse or empty, making alignment across original and synthetic data fragile. RAPID trains a 
single predictive model once and evaluates predictions—often more scalable when quasi-identifiers would generate thousands of cells. Moreover, RAPID naturally provides actionable diagnostics through standard machine learning interpretability tools: feature importance rankings identify which quasi-identifiers drive risk most strongly, partial dependence plots reveal nonlinear relationships, and 
subgroup analyses highlight vulnerable demographic combinations. DiSCO, being table-based, offers transparency but less guidance on how to mitigate risk beyond coarsening keys.

Seventh, DiSCO's results depend critically on quasi-identifier selection and specification choices. Different quasi-identifier sets or category definitions (e.g., coarsening age from 10 bins to 3) can yield substantially different risk assessments for the same data, as cell homogeneity ($p_{stq}=1$) changes with table granularity. Additionally, aligning quasi-identifier levels across original
and synthetic datasets (including unioning categories that appear in one but not the other) introduces further complexity and sensitivity. RAPID sidesteps this fragility by treating quasi-identifiers as model features rather than pre-defined keys: the predictive model automatically learns which variables and interactions matter most for inference, without requiring hand-crafted contingency table 
specifications. Risk estimates remain stable across reasonable quasi-identifier definitions because the model adapts flexibly to the provided feature set.

RAPID addresses these limitations by explicitly modeling the adversary's inference task: training predictive models on synthetic data and evaluating them on original quasi-identifiers. This attack-realistic approach naturally handles continuous targets without discretization, detects risk even when $p_{stq} < 1$ through confidence thresholding, produces individual-level confidence scores enabling 
nuanced risk stratification, aligns with genuine predictive capability rather than synthetic artifacts, calibrates for class imbalance through baseline-normalized scoring, and scales to high-dimensional settings while providing actionable diagnostics. Where DiSCO asks ``does this quasi-identifier group uniquely determine the sensitive value in synthetic data?'' RAPID asks ``could an adversary trained on synthetic data accurately infer this individual's sensitive attribute with 
confidence substantially exceeding baseline?''—directly simulating the inferential disclosure pathway under realistic modeling capabilities and class-aware evaluation.


Case: angreifer augment synthetic data wiht public data.

DiSCO flags records as attribute-disclosive when the synthetic data imply a unique target value for a given quasi-identifier combination $q$ (i.e., the synthetic column proportion $p_{stq}=1$ within that $q$-group), and it then counts the number of original records in those $q$-groups with that target $t$ (hence “Correct Original”). Formally (their Eq. 11),
$\text{DiSCO} = 100 \times \sum_{q}\sum_{t} ( d_{tq} \mid p_{stq}=1)/N_d$,
with related measures like DCAP and TCAP, alignment of $q/t$ levels across GT (original) and SD (synthetic), and options such as denominator limits and grouping continuous targets before cross-tabulation.

Why RAPID can be preferable over DiSCO:

1) Attack-realistic and model-based.
RAPID explicitly trains a predictor on synthetic ($\mathbf{X}^{(s)}, \mathbf{y}^{(s)}$) and evaluates it on original $\mathbf{X}$, mirroring how an intruder would use released synthetic data to infer sensitive attributes. DiSCO is table-based on chosen keys $q$ and does not model $y \mid \mathbf{X}$. This makes RAPID closer to a modern inference attack than a key-uniqueness check.  

2) Works natively for continuous y.
RAPID handles continuous targets via regression and an accuracy/risk threshold (e.g., relative error), avoiding discretization. DiSCO needs binning/grouping of continuous targets before cross-tabulation, and results can be sensitive to the grouping choice.  

3) Individual-level risk, not just group flags.
RAPID produces per-record scores (e.g., relative confidence $r_i$ or accuracy indicators $g_i$) and summary curves over thresholds, enabling granular diagnostics and subgroup auditing. DiSCO is primarily an aggregated share of originals that become disclosive under synthetic $q$-groups.  

4) No fragile dependence on quasi-identifier design.
RAPID can use all non-sensitive features (or any subset) without hand-crafted $q$. DiSCO requires selecting and aligning $q$ across GT/SD (including unioning levels), and results can change markedly with key choice or coarsening.  

5) Captures complex structure.
With flexible learners (RF/GBM/GLM, etc.), RAPID detects nonlinearities and interactions in $y\mid \mathbf{X}$. DiSCO’s logic is based on degenerate synthetic conditional distributions within $q$ (i.e., $p_{stq}=1$), which can miss inferrability arising from complex, high-dimensional relations that don’t show as perfect within-cell certainty.  

6) Better aligned to “true” risk (fewer false alarms from synthetic artifacts).
DiSCO can count records as risky when the synthetic data happen to be degenerate within a q cell—even when the original isn’t uniquely determined by that q (the paper notes a distinct DiSDiO variant that requires $p_{dtq}=1$). RAPID only flags risk when a model trained on SD actually predicts the original y well from original $\mathbf{X}$$, which is closer to a successful attack rather than a synthetic quirk.  

7) Calibrated to class imbalance via a predictive baseline.
For categorical y, RAPID’s relative-confidence score compares the model’s probability for the true class to a marginal baseline estimated over the original $\mathbf{X}$. This is analogous in spirit to synthpop’s baseline scaling (e.g., $baseCAP_d$), but operates directly on predicted probabilities, yielding interpretable “how much better than chance” risk summaries.  

8) Scales in wide/high-cardinality settings.
Building and aligning large contingency tables for many keys or high-cardinality factors (as DiSCO requires) can be heavy. RAPID trains a model once and evaluates predictions—often more scalable when q would explode combinatorially.  

9) Actionable diagnostics.
RAPID naturally provides feature importance, partial-dependence, and subgroup risk profiles—useful for custodians to decide which variables or relations to weaken. DiSCO is transparent and simple, but offers less guidance on how to mitigate beyond coarsening keys.  

When DiSCO is still attractive
	•	Transparent, key-based story for stakeholders comfortable with SDC notions of keys/uniques.
	•	Built-in to synthpop with DCAP/TCAP comparators and options like denominator limits.
	•	Quick categorical screening where clear quasi-identifiers are known a priori.  


Bottom line: \\
DiSCO asks: “Does the synthetic table make a target value appear uniquely determined for this key?” \\
RAPID asks: “Could an attacker trained only on the synthetic data accurately infer my sensitive attribute from non-sensitives?”

For modern inference-attack realism, continuous targets, high-dimensional structure, and record-level diagnostics, RAPID offers clear advantages. For simple, explainable, key-centric screening that integrates with synthpop’s tooling, DiSCO remains useful.  



Critique of the holdout approach (for privacy/risk)
\begin{itemize}
    \item 	Measures memorization, not inference. Its privacy statistic is the fraction of synthetic records that are closer to training than to holdout records (distance-to-closest-record, DCR). A value near 50% suggests no special closeness to training data—good for memorization—but gives no guarantee that sensitive attributes can’t be accurately inferred from quasi-identifiers. High inference risk can coexist with “safe” DCR.  
    \item 	Sensitive to discretization and distance choices. The framework discretizes all variables (with a cap on cardinality) and then uses Hamming (or another) distance. Risk conclusions can vary with binning thresholds, category lumping, and metric choice—especially in mixed-type, high-dimensional data.     \item High-dimensional sparsity issues. Nearest-neighbor distances become unstable as dimensionality grows (curse of dimensionality). Even with binning, “closeness” can be hard to interpret and may over- or under-state record proximity.  
    \item Requires internal holdout access. Proper evaluation needs a real holdout split of the original data. That’s fine for an internal steward, but it doesn’t mirror what an external attacker can do with only the released synthetic file.  
    \item Utility–risk conflation gaps. The holdout framework’s fidelity side (TVDs of k-way marginals) is useful for representativeness, but it’s orthogonal to attribute-inference risk: a synthesizer can pass fidelity checks yet still leak labels via strong predictive relations; conversely, failing a DCR test doesn’t quantify per-variable harm.  
    \item Potential to be gamed by post-processing. Small perturbations or aggressive discretization can inflate DCR to look “safer” without materially reducing the ability to infer sensitive attributes.
\end{itemize}

Advantages of RAPID over holdout-based evaluation
\begin{itemize}
    \item 	Directly measures attribute-inference risk. RAPID asks the attacker’s question (“Can I correctly infer a confidential attribute from quasi-identifiers?”) by training on the synthetic file and scoring on the real covariates. The holdout method instead assesses (i) fidelity via marginal distribution distances and (ii) privacy via nearest-neighbor closeness to training vs. holdout records; it does not quantify how well an adversary could predict the confidential label.  
    \item 	Interpretable, policy-ready metric. RAPID yields a single, bounded risk rate—e.g., the share of people for whom an attacker is ≥τ times more confident than the model’s class-baseline (categorical) or within ε relative error (continuous). This is easier to explain than distributions of total-variation distances or nearest-neighbor gaps.  
    \item 	Base-rate calibration. RAPID’s class-normalized confidence (“relative confidence”) avoids spurious risk inflation under class imbalance or miscalibration—issues common when using raw probabilities or accuracy. The holdout framework is model-free and therefore cannot calibrate risk to class prevalences or prediction confidence.  
    \item 	Aligned attacker model. RAPID constrains the intruder to the released synthetic file (train-on-synthetic, test-on-original), matching real public-release settings. The holdout approach requires access to an internal holdout of the original data during assessment and targets memorization/overfitting rather than inferential disclosure.  
    \item 	Comparable across releases and synthesizers. Because RAPID normalizes confidence by a model-implied baseline and reports a thresholded rate, numbers remain comparable even when marginal class distributions or calibration differ across releases; holdout TVDs and nearest-neighbor shares can shift with discretization, binning caps, and sample splits. 
\end{itemize}

Bottom line. Holdout-based evaluation is valuable for fidelity and for detecting memorization/overfitting to training records in a model-free way. RAPID complements—and for disclosure risk, improves upon—this by quantifying exactly what many stewards and regulators care about: how often a capable intruder, using only the released synthetic file, could be confidently correct about a person’s confidential attribute, with class-imbalance calibration and a single interpretable rate. 


Public Use Files (PUFs) are datasets released openly, often on public websites, with minimal barriers to access. These files are heavily anonymized through traditional Statistical Disclosure Control (SDC) techniques such as generalization, suppression, and top/bottom coding. The emphasis is on eliminating both direct identifiers and key indirect identifiers to ensure that the risk of reidentification is extremely low. While some advocate for the use of DP to strengthen protections here, it's important to recognize that DP is not inherently designed for static data release. Applying DP to create “DP-anonymized” PUFs (e.g., DP-synthetic data) is technically possible but offers limited practical utility unless access is tightly controlled and a privacy budget is enforced, which contradicts the very openness of PUFs.

Scientific Use Files (SUFs), by contrast, are made available only under formal agreements, typically to accredited researchers with a legitimate need. These datasets are usually more detailed than PUFs and may retain some potentially identifying indirect variables. Though still anonymized, SUFs acknowledge the residual risk of linkage attacks, especially when external datasets are available. While DP could theoretically be applied to SUFs to enhance protections, its formal guarantees only hold under carefully controlled, bounded interaction models—not static release. Hence, using DP here may give a false sense of security, especially if data are copied or used repeatedly in analysis pipelines without strict budget enforcement.

RM: possible limitations:
\begin{itemize}
  \item RAPID is not model-agnostic, i.e., the risk measure is strongly dependent on the fitted model. This also includes feature engineering. A sufficiently motivated attacker could derive new features from existing variables, potentially boosting prediction performance substantially. RAPID is therefore not a worst-case, conservative measure like the Bayesian risk, which provides an upper bound on risk. Instead, in comparison with Bayesian risk (Reiter, 2014) it can be interpreted as a lower-bound estimate: the true risk will be at least this large.
  \item What if there are multiple sensitive variables, which is basically always the case in practice?
  \item A comparison with bayesian risk measures for synthetic data could also be interesting, see:
  \begin{itemize}
     \item Bayesian Estimation of Attribute and Identification  Disclosure Risks in Synthetic Data (Hu, 2021)
     \item Bayesian Estimation of Disclosure Risks for  Multiply Imputed, Synthetic Data (Reiter et al., 2014)
     \item Implementation as R package: Bayesian Estimation of Attribute Disclosure Risks in Synthetic Data with the AttributeRiskCalculation R Package (Hornby & Hu, 2022) 
   \end{itemize}
\end{itemize}

\subsection{Ethical and legal considerations}\label{subsec:ethics}
The evaluation assumes the attacker does \emph{not} access original labels beyond those implicit in \(\mathbf{Z}\) for scoring. Risk numbers should be interpreted in the context of legal thresholds and harm models for the specific domain (e.g., health, labor). When reporting, avoid releasing per-record scores \(r_i\) or \(h_i\); publish only aggregated metrics and confidence intervals.

\subsection{Summary}\label{subsec:summary}
The proposed workflow---train on synthetic, score on real covariates, normalize by model-implied baselines, and summarize by \(\mathrm{RCIR}\) with uncertainty---yields an interpretable, model-agnostic measure of inference disclosure risk. The real data illustration shows how to operationalize the protocol on a standard benchmark; the same steps apply to domain datasets (e.g., longitudinal panels or mobility traces) with choice of \(\tau/\varepsilon\) tailored to sensitivity.

\cite{pilgram25} erwaehne, dass es dazu passt als Mosaikstein.

\section{conclusion}

We introduced a calibrated, attacker-realistic measure of inference disclosure for synthetic microdata. By normalizing correct-class confidence to a class-specific baseline and summarizing the share above a threshold, the metric is interpretable, robust to class imbalance, and easily integrated with existing synthetic-data workflows. It complements utility diagnostics and aligns with modern intruder models from both official statistics and machine learning. Future work includes formal power analyses for threshold selection, extensions to multi-label outcomes, and integration with end-to-end risk-utility dashboards.

\color{black}

\bibliographystyle{plainnat}  
\bibliography{references}  

\appendix
\section{Simulation Data Generation}
\label{app:datagen}

We simulate $n$ independent microdata records with six variables: gender ($G$), age ($A$), education ($E$), income ($I$), health score ($H$), and disease status ($D$). The design encodes realistic, policy-relevant dependencies through a latent socioeconomic status (SES) variable while remaining transparent and tunable via a global dependency parameter $\kappa\ge 0$. Throughout, $\mathsf{TN}(\mu,\sigma^2;[a,b])$ denotes a normal distribution truncated to $[a,b]$, and we use standardized predictors $\tilde{X}=(X-\mu_X)/\sigma_X$ when indicated.

\paragraph{Dependency mechanism.}
All variable dependencies flow through a shared latent variable:
\[
\text{SES}_i \sim \mathcal{N}(0,1),
\]
representing unobserved socioeconomic status. The strength of dependencies is controlled by signal and noise weights derived from $\kappa$:
\[
w_{\text{signal}} = \sqrt{\frac{\kappa}{1+\kappa}}, \qquad
w_{\text{noise}} = \sqrt{\frac{1}{1+\kappa}}.
\]
At $\kappa=0$, relationships are driven purely by noise ($w_{\text{signal}}=0$, $w_{\text{noise}}=1$). As $\kappa\to\infty$, dependencies become deterministic ($w_{\text{signal}}\to 1$, $w_{\text{noise}}\to 0$). At the default $\kappa=1$, signal and noise contribute equally ($w_{\text{signal}}=w_{\text{noise}}=1/\sqrt{2}$), yielding approximately 50\% explained variance.

\paragraph{Age.}
We draw age from a truncated normal to reflect adult populations:
\[
A_i \sim \mathsf{TN}(\mu_A,\sigma_A^2;[a_{\min},a_{\max}]), 
\quad \text{defaults: } \mu_A=45,\ \sigma_A=12,\ a_{\min}=18,\ a_{\max}=85.
\]

\paragraph{Education.}
Education is an ordinal categorical variable with three levels $\{0,1,2\}$ corresponding to \emph{low}, \emph{medium}, and \emph{high} attainment. We generate $E_i$ from a latent continuous variable that depends on both SES and age:
\[
L_i = w_{\text{signal}} \cdot (0.8\cdot\text{SES}_i - 0.4\cdot\tilde{A}_i) + w_{\text{noise}}\cdot\varepsilon_i,
\quad \varepsilon_i\sim\mathcal{N}(0,1),
\]
\[
E_i =
\begin{cases}
0 & \text{if } L_i < c_1, \\
1 & \text{if } c_1 \le L_i < c_2, \\
2 & \text{if } L_i \ge c_2,
\end{cases}
\quad c_1 = -0.3,\ c_2 = 0.7.
\]
The negative age coefficient reflects the empirical pattern of younger cohorts having higher educational attainment.

\paragraph{Income.}
We generate log-income using a linear model:
\[
\log I_i^\star = w_{\text{signal}}\cdot(0.5\cdot\text{SES}_i + 0.3\cdot\tilde{A}_i + 0.25\cdot E_i) + w_{\text{noise}}\cdot\varepsilon_i,
\quad \varepsilon_i\sim\mathcal{N}(0,1),
\]
\[
I_i = \exp(10 + \log I_i^\star).
\]
The constant 10 centers income around realistic values (approximately \$20,000--\$40,000).

\paragraph{Health score.}
$H$ is a continuous health measure on $[0,100]$ created via sigmoid transformation of a linear predictor:
\[
H_i^\star = w_{\text{signal}}\cdot\bigl(0.6\cdot\text{SES}_i - 0.5\cdot\tilde{A}_i + 0.2\cdot E_i + 0.2\cdot\widetilde{\log I_i}\bigr) + w_{\text{noise}}\cdot\zeta_i,
\quad \zeta_i\sim\mathcal{N}(0,1),
\]
\[
H_i = \frac{100}{1 + \exp(-H_i^\star)}.
\]
The negative age coefficient reflects declining health with age, while positive SES, education, and income coefficients represent protective effects.

\paragraph{Disease status.}
$D_i\in\{\texttt{healthy},\texttt{diabetic},\texttt{hypertensive}\}$ is sampled via a multinomial logit with \texttt{healthy} as the baseline. Unlike other variables, disease dependencies scale \emph{linearly} with $\kappa$ rather than through signal/noise weights:
\begin{align*}
\log\frac{\Pr(D_i=\texttt{diabetic})}{\Pr(D_i=\texttt{healthy})}
&= -1.5 + \kappa\cdot\bigl(0.8\cdot\tilde{A}_i - 0.3\cdot\widetilde{\log I_i} - 0.2\cdot E_i\bigr),\\
\log\frac{\Pr(D_i=\texttt{hypertensive})}{\Pr(D_i=\texttt{healthy})}
&= -1.3 + \kappa\cdot\bigl(1.0\cdot\tilde{A}_i - 0.2\cdot\widetilde{\log I_i} - 0.1\cdot E_i\bigr).
\end{align*}
This linear scaling creates stronger dependency effects at high $\kappa$: older age raises risk, while higher income and education are mildly protective. The fixed intercepts ($-1.5$ and $-1.3$) induce a baseline class imbalance favoring \texttt{healthy} outcomes.

\paragraph{Gender.}
Gender is binary with mild dependency on SES, age, and education:
\[
\eta_i = w_{\text{signal}}\cdot(0.3\cdot\text{SES}_i - 0.2\cdot\tilde{A}_i + 0.2\cdot E_i),
\]
\[
G_i \sim \text{Bernoulli}\bigl(\text{logit}^{-1}(\eta_i)\bigr),
\quad G_i\in\{\texttt{female}, \texttt{male}\}.
\]

\paragraph{Controlling dependence strength.}
The global parameter $\kappa\ge 0$ controls the strength of all dependencies:
\begin{itemize}
\item $\kappa=0$: Variables retain weak dependencies due to fixed intercepts in the disease model, but signal contributions vanish ($w_{\text{signal}}=0$).
\item $\kappa=1$ (default): Balanced signal-to-noise ratio, yielding moderate dependencies.
\item $\kappa\gg 1$: Near-deterministic relationships as $w_{\text{signal}}\to 1$ and disease coefficients grow large.
\end{itemize}
Because disease logits scale linearly with $\kappa$ while other variables use the signal-to-noise transformation, disease dependencies strengthen more rapidly at high $\kappa$.

\paragraph{Defaults and realism.}
With default $\kappa=1$, the simulation produces realistic marginal distributions and moderate dependencies: income rises with SES, age, and education; health declines with age but improves with socioeconomic status; and the probabilities of \texttt{diabetic} and \texttt{hypertensive} increase with age and decrease with income and education. These defaults can be adapted to domain-specific baselines by adjusting variable-specific parameters without changing $\kappa$.


\end{document}
